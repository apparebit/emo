% \iffalse meta-comment
%
% emo•ji for all (LaTeX engines)
% (C) Copyright 2023 by Robert Grimm
% Released under LPPL v1.3c or later
% <https://github.com/apparebit/emo>
%
% \fi
% ^^A ----------------------------------------------------------------------------------
% \iffalse
%<*scaffold>
\iffalse
%</scaffold>
% ======================================================================================
%<*readme>
# emo•ji for all (LaTeX engines)

This package defines the `\emo{<emoji-name>}` macro for including color emoji in
a document no matter the LaTeX engine. It uses the Noto color emoji font if the
engine supports doing so and falls back onto PDF graphics otherwise. In either
case, `\emo{desert-island}` results in 🏝 and `\emo{parrot}` results in 🦜. Emo
may come in particularly handy when dealing with academic publishers that
provide only minimal support for non-Latin scripts (cough,
[ACM](https://www.acm.org), cough).

Emo's source repository is <https://github.com/apparebit/emo>. It also is
available [through CTAN](https://ctan.org/pkg/emo). Emo supports conversion to
HTML with [LaTeXML](https://github.com/brucemiller/LaTeXML) or
[TeX4ht](https://tug.org/tex4ht/). When using the latter tool, be sure to use
|make4ht -l| as invocation.


## Package Options

When emo is used with the `extra` option, this package also defines the
`\lingchi` and `\YHWH` macros for 凌遲 and יהוה, respectively. Both macros
preserve a subsequent space as space, no backslash needed.

When used with the `index` option, this package also emits a raw index entry for
each use of an emoji into an emo index or `.edx` file.


## Installation

To **extract files** embedded in [emo.dtx](emo.dtx), run `pdftex emo.dtx`. Note
that plain old `tex` won't do, since it mangles this README. `pdflatex` works,
but also generates the package documentation. The embedded files are `build.sh`,
`emo.ins`, `emo.sty`, `emo.sty.ltxml`, and `README.md`.

To **build the documentation** embedded in `emo.dtx`, run `source build.sh`. The
shell script invokes `pdflatex emo.dtx` thrice and `makeindex` once each for the
change and the symbol indices, resulting in [emo.pdf](emo.pdf).

To **configure the emoji**, run `python3 config/emo.py` with appropriate
arguments. The [package documentation](emo.pdf) explains the configuration tool
in detail, but you may find the `-h` for help option sufficient to get started.

To **install this package**, place `emo.def`, `emo.sty`, `emo.sty.ltxml`,
`emo-lingchi.ttf`, and the `emo-graphics` directory with the fallback PDF files
somewhere where LaTeX can find them. In a pinch, your project directory will do.


## Supported Emoji

By default, emo supports only a few emoji:

1️⃣ ☣️ ⚖️ ✔️ ➕ 🇪🇺 🉐 🌁 🌍 🏛️ 🏝️ 🏟️ 🏳️‍🌈 🏷️ 👁️ 👥 💥 💱 💾 📈 📐 📟 🔍
🕵️ 🗑️ 😡 🛑 🤖 🤝 🤦 🤯 🦜 🧑‍⚖️ 🧻 🧾

Their names are keycap-one, biohazard, balance-scale, check-mark, plus, eu,
japanese-bargain-button, foggy, globe-africa-europe, classical-building,
desert-island, stadium, rainbow-flag, label, eye, busts, collision,
currency-exchange, floppy-disk, chart-increasing, triangular-ruler, pager,
loupe-left, detective, wastebasket, enraged-face, stop-sign, robot, handshake,
person-facepalming, exploding-head, parrot, judge, roll-of-paper, and receipt.

The [package's documentation](emo.pdf) explains the underlying naming scheme and
also how to reconfigure which emoji are supported. The [emo.py](config/emo.py)
script takes care of the heavy lifting during reconfiguration, converting SVG
into PDF files and generating an updated `emo.def` file.


## Copyright and Licensing

This package combines code written in LaTeX, Python, and Perl with Unicode data
about emoji as well as graphics and fonts derived from Google's Noto fonts. As a
result, a number of different licenses apply, all of which are [OSI
approved](https://opensource.org/licenses/) and non-copyleft:

  * This package's [LaTeX code](emo.dtx) is © Copyright 2023 by Robert Grimm and
    has been released under the [LPPL
    v1.3c](https://www.latex-project.org/lppl/lppl-1-3c/) or later.
  * The [emo.py](config/emo.py) configuration script also is © Copyright 2023 by
    Robert Grimm but has been released under the [Apache 2.0
    license](https://www.apache.org/licenses/LICENSE-2.0).
  * The [emoji-test.txt](config/emoji-test.txt) configuration file is a data
    file from [Unicode TR-51](https://unicode.org/reports/tr51/) and hence
    subject to the [Unicode License](https://www.unicode.org/license.txt).
  * The `emo-lingchi.ttf` font is a two-glyph subset of the serif traditional
    Chinese version of Google's [Noto
    fonts](https://github.com/notofonts/noto-cjk) and hence subject to the [SIL
    Open Font License v1.1](https://scripts.sil.org/ofl).
  * The PDF graphics in the `emo-graphics` directory are derived from the
    sources for [Noto's color emoji](https://github.com/googlefonts/noto-emoji)
    and hence subject to the Apache 2.0 license.

%</readme>
% --------------------------------------------------------------------------------------
%<*buildscript>
pdflatex emo.dtx
makeindex -s gind.ist -o emo.ind emo.idx
makeindex -s gglo.ist -o emo.gls emo.glo
pdflatex emo.dtx
pdflatex emo.dtx
%</buildscript>
% ======================================================================================
%<*scaffold>
\fi
\def\nameofplainTeX{plain}
\ifx\fmtname\nameofplainTeX\else
    \expandafter\begingroup
\fi
%</scaffold>
% --------------------------------------------------------------------------------------
%<*install>
\input docstrip.tex
\keepsilent
\askforoverwritefalse
\preamble

emo•ji for all (LaTeX engines)
(C) Copyright 2023 by Robert Grimm
Released under LPPL v1.3c or later
<https://github.com/apparebit/emo>

\endpreamble
\usedir{tex/latex/emo}
\generate{
    \file{\jobname.sty}{\from{\jobname.dtx}{package}}}
\generate{
    \nopreamble\nopostamble
    \file{\jobname.sty.ltxml}{\from{\jobname.dtx}{latexml-binding}}}
%</install>
%<install>\endbatchfile
% --------------------------------------------------------------------------------------
%<*scaffold>
\usedir{source/latex/emo}
\generate{\file{\jobname.ins}{\from{\jobname.dtx}{install}}}
\nopreamble\nopostamble
\usedir{source/latex/emo}
\generate{\file{build.sh}{\from{\jobname.dtx}{buildscript}}}
\usedir{doc/latex/emo}
\generate{\file{README.md}{\from{\jobname.dtx}{readme}}}
\ifx\fmtname\nameofplainTeX
    \expandafter\endbatchfile
\else
    \expandafter\endgroup
\fi
%</scaffold>
% ======================================================================================
% See https://tug.org/TUGboat/tb29-2/tb92pakin.pdf.
%<*scaffold>
\ProvidesFile{emo.dtx}
%</scaffold>
%<package>\NeedsTeXFormat{LaTeX2e}
%<package>\ProvidesPackage{emo}
%<*scaffold,package>
    [2023/04/21 v0.3 emo•ji for all (LaTeX engines)]
%</scaffold,package>
% ======================================================================================
%<*driver>
\PassOptionsToPackage{utf8}{inputenc}
\documentclass{ltxdoc}
% Override the default \small, which looks odd typeset in Inconsolata.
\renewcommand{\MacroFont}{\normalsize\ttfamily}
\usepackage[extra]{\jobname}
\usepackage{enumitem}
\usepackage{parskip}
\usepackage{inconsolata}

% Use BlackFoundry's Inria <https://tug.org/FontCatalogue/inriaserifregular/>
\usepackage[T1]{fontenc}
\usepackage[lining]{InriaSerif}
\renewcommand*\oldstylenums[1]{{\fontfamily{InriaSerif-OsF}\selectfont #1}}
\let\oldnormalfont\normalfont
\def\normalfont{\oldnormalfont\mdseries}

\usepackage{xcolor}
\usepackage{hyperref}
\definecolor{spot}{HTML}{353598}
\hypersetup{allcolors=spot}
\EnableCrossrefs
\CodelineIndex
\RecordChanges
\begin{document}
    \DocInput{\jobname.dtx}
\end{document}
%</driver>
% ======================================================================================
% \fi
%
% \changes{0.1}{}{Initial release}
% \changes{0.2}{}{Prefix PDF and font files with ``emo-''}
% \changes{0.2}{}{Support pdftex for extracting emo.dtx}
% \changes{0.3}{}{Support conversion to HTML with TeX4ht}
%
% \GetFileInfo{\jobname.dtx}
%
% \DoNotIndex{\{,\},\x,\\}
% \DoNotIndex{\begingroup,\char,\csname,\def}
% \DoNotIndex{\else,\endcsname,\endgroup,\expandafter}
% \DoNotIndex{\fi,\ifcsname,\ifluatex,\index}
% \DoNotIndex{\newcommand,\newif,\newindex}
% \DoNotIndex{\relax,\string,\textbackslash,\textsf,\texttt}
%
% \newlist{inlinenum}{enumerate*}{1}
% \setlist[inlinenum]{label=(\alph{inlinenumi})}
% \newenvironment{verbatimish}{%
%     \ttfamily%
%     \obeylines%
%     \obeyspaces%
%     \vspace{\the\parskip}%
%     \setlength{\parskip}{0pt}%
%     \setlength{\parindent}{1em}%
% }{}
%
% \title{emo•ji for all\\(LaTeX engines)}
% \author{\href{https://apparebit.com}{Robert Grimm}}
% \date{Version \fileversion\ (\filedate)}
%
% \maketitle
%
% \begin{abstract}
% \noindent{}Emo implements the |\emo|\marg{emoji-name} command for including
% color emoji such as |\emo{desert-island}| for \emo{desert-island} or
% |\emo{parrot}| for \emo{parrot} in your documents independent of LaTeX engine.
% The implementation uses the Noto color emoji font if the engine supports it
% and includes PDF graphics otherwise. It also supports conversion to HTML
% either with LaTeXML or TeX4ht. PDF graphics are automatically derived from
% Noto's SVG sources, so the visual appearance is very similar. The source
% repository is at \url{https://github.com/apparebit/emo}. Emo may come in
% particularly handy when dealing with academic publishers that provide only
% minimal support for non-Latin scripts (cough,
% \href{https://authors.acm.org/proceedings/production-information/accepted-latex-packages}{ACM},
% cough).
% \end{abstract}
%
% \tableofcontents
%
%
% ^^A ==================================================================================
% \section{Installation}
%
% The emo package is available through its
% \href{https://github.com/apparebit/emo}{source repository} or through
% \href{https://ctan.org/pkg/emo}{CTAN}. Installation is fairly
% straightforward, though it does involve a lot more files than usual.
%
% \begin{enumerate}
% \item Start by extracting this package's files from |emo.dtx| by running:
% \begin{verbatimish}
%     \$ pdftex emo.dtx
% \end{verbatimish}
% Do \emph{not} use |tex|; it mangles the embedded |README.md|. |pdflatex| also
% extracts the files and then builds the documentation. Embedded files are
% |build.sh|, |emo.ins|, |emo.sty|, |emo.sty.ltxml|, and |README.md|. Extraction
% will overwrite existing files with the same name without asking.
%
% \item Build the package documentation with change and symbol indices by
% running:
% \begin{verbatimish}
%      \$ source build.sh
% \end{verbatimish}
% The shell script invokes |pdflatex emo.dtx| thrice and |makeindex| once each
% for the change and symbol indices to produce |emo.pdf|.
%
% \item Get started reconfiguring supported emoji by running:
% \begin{verbatimish}
% \$ python config/emo.py -h
%     \end{verbatimish}
% For more detailed instructions, see \S\ref{sec:config} below.
%
% \item Put the following files somewhere LaTeX can find them. In a pinch, your
% current project's directory will do. However, emo's installation potentially
% comprises thousands of files. So, you probably want to use a dedicated
% directory and add that to the search path for LaTeX, e.g., by setting the
% |TEXINPUTS| environment variable.
% \begin{enumerate}
% \item |emo.sty| with the package implementation;
% \item |emo.sty.ltxml| with the binding for
%     \href{https://github.com/brucemiller/LaTeXML}{LaTeXML};
% \item |emo.def| with the emoji table;
% \item |emo-lingchi.ttf| with the two glyphs for |\lingchi|;
% \item |emo-graphics| with the fallback PDF graphics.
% \end{enumerate}
% TeX Live requires that each package's files have unique names. For that
% reason, the PDF graphics in the |emo-graphics| directory start with the |emo-|
% prefix as well.
%
% \end{enumerate}
%
% When running on the LuaLaTeX engine, the emo package also uses the Noto color
% emoji (|NotoColorEmoji.ttf|) and Linux Libertine (|LinLibertine_R.otf|) fonts,
% with the latter used for rendering |\YHWH| only. Neither file is included with
% emo's distribution, since both of them are distributed with major TeX
% distributions already. If they are not included with your LaTeX distribution,
% you can find them on CTAN. The |emo-lingchi.ttf| font distributed with emo is
% a two glyph subset of |NotoSerifTC-Regular.otf|, i.e., the traditional Chinese
% version of Noto serif.
%
%
% ^^A ==================================================================================
% \section{Usage}
%
% As usual, you declare your document's dependency on emo with
% |\usepackage{emo}|. In addition to the unadorned form, emo takes up to two
% options:
% \begin{description}
% \item[extra] Also define the |\lingchi| and |\YHWH| macros, which produce
%     \lingchi and \YHWH.
% \item[index] Create an emoji index tagged |emo| with the |.edx| extension for
%     the raw index and the |.end| extension for the processed index. This
%     option relies on the |index| package, generates the raw |.edx| file,
%     but does not build or use the processed index.
% \end{description}
%
% \DescribeMacro{\emo}
% An |\emo|\marg{emoji-name} invocation expands to the named emoji. For
% LuaLaTeX, it uses the Noto color emoji font. For all other engines, it uses
% PDF graphics. That way, |\emo{desert-island}| results in~\emo{desert-island}
% and |\emo{parrot}| results in~\emo{parrot}.
%
% Since LaTeX tends to produce a lot of command line noise about underfull boxes
% and loaded fonts, it's a easy to miss meaningful warnings. For that reason,
% |\emo| expands to an attention-seeking error message upon undefined emoji
% names. For example, |\emo{boo}| produces \emo{boo}.
%
%
% ^^A ----------------------------------------------------------------------------------
% \subsection{Emoji Names}
%
% With some exceptions, emo's names for emoji are automatically derived from
% their Unicode names, with letters converted to lowercase, punctuation such as
% commas, colons, quotes, and parentheses stripped, and interword spaces
% replaced by dashes. Furthermore, instead of the rather verbose
% |dark-skin-tone|, |medium-dark-skin-tone|, etc modifiers, emo
% uses the more succinct |darkest|, |darker|, |medium|, |lighter|, and
% |lightest|.
%
% For some emoji names, emo goes further by hard-coding shorter names. Those
% names are listed in Table~\ref{tab:special-names}.
%
% \begin{table}
% \caption{Exceptional emoji names}
% \label{tab:special-names}
% \small\vspace{1em}
% \begin{tabular}{ll}
% \textbf{Transformed Unicode Name} & \textbf{Emo Replacement Name} \\ \hline
% |a-button-blood-type| & |a-button| \\
% |ab-button-blood-type| & |ab-button| \\
% |b-button-blood-type| & |b-button| \\
% |o-button-blood-type| & |o-button| \\
% |bust-in-silhouette| & |bust| \\
% |busts-in-silhouette| & |busts| \\
% |flag-european-union| & |eu| \\
% |globe-showing-americas| & |globe-americas| \\
% |globe-showing-asia-australia| & |globe-asia-australia| \\
% |globe-showing-europe-africa| & |globe-africa-europe| \\
% |hear-no-evil-monkey| & |hear-no-evil| \\
% |index-pointing-at-the-viewer| & |index-pointing-at-viewer| \\
% |index-pointing-at-the-viewer-darkest| & |index-pointing-at-viewer-darkest| \\
% |index-pointing-at-the-viewer-darker| & |index-pointing-at-viewer-darker| \\
% |index-pointing-at-the-viewer-medium| & |index-pointing-at-viewer-medium| \\
% |index-pointing-at-the-viewer-lighter| & |index-pointing-at-viewer-lighter| \\
% |index-pointing-at-the-viewer-lightest| & |index-pointing-at-viewer-lightest| \\
% |keycap-*| & |keycap-star| \\
% |keycap-#| & |keycap-hash| \\
% |keycap-0| & |keycap-zero| \\
% |keycap-1| & |keycap-one| \\
% |keycap-2| & |keycap-two| \\
% |keycap-3| & |keycap-three| \\
% |keycap-4| & |keycap-four| \\
% |keycap-5| & |keycap-five| \\
% |keycap-6| & |keycap-six| \\
% |keycap-7| & |keycap-seven| \\
% |keycap-8| & |keycap-eight| \\
% |keycap-9| & |keycap-nine| \\
% |keycap-10| & |keycap-ten| \\
% |magnifying-glass-tilted-left| & |loupe-left| \\
% |magnifying-glass-tilted-right| & |loupe-right| \\
% |palm-down-hand| & |palm-down| \\
% |palm-down-hand-darkest| & |palm-down-darkest| \\
% |palm-down-hand-darker| & |palm-down-darker| \\
% |palm-down-hand-medium| & |palm-down-medium| \\
% |palm-down-hand-lighter| & |palm-down-lighter| \\
% |palm-down-hand-lightest| & |palm-down-lightest| \\
% |palm-up-hand| & |palm-up| \\
% |palm-up-hand-darkest| & |palm-up-darkest| \\
% |palm-up-hand-darker| & |palm-up-darker| \\
% |palm-up-hand-medium| & |palm-up-medium| \\
% |palm-up-hand-lighter| & |palm-up-lighter| \\
% |palm-up-hand-lightest| & |palm-up-lightest| \\
% |rolling-on-the-floor-laughing| & |rofl| \\
% |see-no-evil-monkey| & |see-no-evil| \\
% |speak-no-evil-monkey| & |speak-no-evil| \\
% \end{tabular}
% \end{table}
%
% Emo's |emo.def| contains the names and codepoints of all currently supported
% emoji. Its distribution also includes the |emoji-test.txt| file, which is part
% of \href{https://unicode.org/reports/tr51/}{Unicode TR-51} and contains the
% names and codepoints of all \emph{potentially} supported emoji, i.e., all
% emoji. It further organizes emoji into groups and subgroups, with the current
% (sub)group being the one named on the closest line above the emoji that starts
% with |# |(|sub|)|group:|. As described in the next section, the group and
% subgroup names can be used during configuration for concisely naming a large
% number of emoji.
%
%
% ^^A ----------------------------------------------------------------------------------
% \subsection{Extras}
%
% \DescribeMacro{\lingchi}
% \DescribeMacro{\YHWH}
% The |\lingchi| and |\YHWH| macros take no arguments and produce \lingchi and
% \YHWH. They are only available if emo is used with the \textsf{extra} option.
% The former renders the Chinese term for ``death by a thousand cuts.'' While
% originally an execution method, the term applies to surprisingly many software
% systems as well. The latter produces the Tetragrammaton, the Hebrew name for
% God. Observant Jews never utter what's written, not even in their thoughts,
% substituting Adonai (``My Lord''), Elohim (``God''), or HaShem (``The Name'')
% instead. In my mind, that nicely mirrors the very incomprehensibility of
% \YHWH. Both macros preserve a subsequent space as space, no backslash needed.
%
%
% ^^A ----------------------------------------------------------------------------------
% \subsection{Conversion to HTML}
%
% Emo supports conversion to HTML with either
% \href{https://github.com/brucemiller/LaTeXML}{LaTeXML} or
% \href{https://tug.org/tex4ht/}{TeX4ht}. The former is implemented by a
% separate ``binding'' against LaTeXML's Perl API. The latter is implemented by
% the emo package itself; it requires processing with LuaLaTeX, e.g., by passing
% |-l| to the |make4ht| tool.
%
%
% ^^A ==================================================================================
% \section{Configuration}
% \label{sec:config}
%
% Emo's implementation is actually split over two files: |emo.sty| is extracted
% from |emo.dtx| and defines the substance of the package, its options, its
% helper macros, and the user-visible |\emo|, |\lingchi|, and |\YHWH| macros.
% Currently supported emoji are defined by the emoji table in the second file,
% |emo.def|. For every supported emoji, the file contains a command
% |\emo@emoji@|\meta{emoji-name} with the emoji's codepoints as value.
%
% Configuration automates the regeneration of the emoji table for arbitrary
% numbers of emoji. |config/emo.py| is the script and |config/emoji-test.txt| is
% the list of all emoji from the Unicode standard.
%
%
% ^^A ----------------------------------------------------------------------------------
% \subsection{Running the Configuration Script}
%
% To update emo's configuration, invoke the |config/emo.py| script:
% \begin{verbatimish}
%     \$ python3 config/emo.py \meta{selector} \meta{selector} $\ldots$
% \end{verbatimish}
%
% Each selector may be:
% \begin{itemize}
% \item The literal |ALL| (case-sensitive) for \emph{all} emoji.
% \item Name of a group in |emoji-test.txt| lowercased and with spaces replaced
%     by dashes and ampersand |&| replaced by an |and|; e.g.,
%     |travel-and-places|.
% \item Name of a group, a double colon |::|, and name of a subgroup, again
%     lowercased and with spaces replaced by dashes and |&| by an |and|; e.g.,
%     |travel-and-places::place-geographic|.
% \item The name of an emoji; e.g., |desert-island|.
% \end{itemize}
% For conjunctive group names, such as ``Smileys \& Emotion'' (|emoji-test.txt|)
% or ``smileys-and-emotion'' (|emo.py|), the configuration script also accepts
% either of the two nouns as a shortcut, e.g., ``smileys'' or ``emotion.''
%
% For data safety, |emo.py| does not overwrite PDF graphics and hence can only
% \emph{add} emoji to the configuration. To remove emoji, simply remove their
% PDF graphics from |emo-graphics| and then run |emo.py| without selector
% arguments, which updates the emoji table accordingly.
%
% |emo.py| effectively treats |emoji-test.txt| as registry of all emoji and the
% filenames of PDF graphics in |emo-graphics| as emo's current inventory. For
% all emoji named by selector arguments but not in the inventory, |emo.py|
% converts the SVG source graphic from the Noto color emoji sources to a PDF
% file and deletes the |/Page| |/Group| object from the the PDF again, since
% that object trips up |pdflatex|. And yeah, |emo.py| automatically downloads
% the Noto color emoji sources if necessary.
%
%
% ^^A ==================================================================================
% \section{Copyright and Licensing}
%
% Since emo's distribution includes not only LaTeX code but also a substantial
% Python script, Unicode data about emoji, as well as graphics and fonts derived
% from Google's Noto project, a number of different licenses apply. All of them
% are \href{https://opensource.org/licenses/}{OSI approved} and non-copyleft:
% \begin{itemize}
% \item This package's LaTeX and also Perl code extracted from |emo.dtx|
%     is © Copyright 2023 by Robert Grimm and has been released
%     under the \href{https://www.latex-project.org/lppl/lppl-1-3c/}{LPPL v1.3c}
%     or later.
% \item The |config/emo.py| script also is © Copyright 2023 by Robert Grimm but
%     has been released under the
%     \href{https://www.apache.org/licenses/LICENSE-2.0}{Apache 2.0 license}.
% \item The [config/emoji-test.txt] configuration file is a data file from
%     \href{https://unicode.org/reports/tr51/}{Unicode TR-51} and hence subject
%     to the \href{https://www.unicode.org/license.txt}{Unicode License}.
% \item The |emo-lingchi.ttf| font is a two-glyph subset of the traditional
%     Chinese version of Google's
%     \href{https://github.com/notofonts/noto-cjk}{Noto serif} and hence subject
%     to the \href{https://scripts.sil.org/ofl}{SIL Open Font License v1.1}.
% \item The PDF graphics in the |emo-graphics| directory are derived from the
%     sources for \href{https://github.com/googlefonts/noto-emoji}{Noto's color
%     emoji} and hence subject to the Apache 2.0 license.
% \end{itemize}
%
%
% \StopEventually{^^A
%     \PrintChanges^^A
%     \setcounter{IndexColumns}{2}^^A
%     \columnsep = 20pt^^A
%     \PrintIndex}
%
%
% ^^A ==================================================================================
% \section{Implementation}
%
% Let's get started with emo's implementation:
%    \begin{macrocode}
%<*package>
%    \end{macrocode}
%
% Except, the package implementation started near the top of the |emo.dtx| file,
% before the documentation preamble. We repeat it here for completeness:
%
% \begin{verbatimish}
% |\NeedsTeXFormat{LaTeX2e}|
% |\ProvidesPackage{emo}|
% |    |[\filedate\space\fileversion\space\fileinfo]
% \end{verbatimish}
%
% And no, I didn't repeat the version number, date, or package information.
% Check |emo.dtx|.
%
%
% ^^A ----------------------------------------------------------------------------------
% \subsection{Package Options}
%
% \begin{macro}{\ifemo@extra}
% \begin{macro}{\ifemo@index}
% Emo's \textsf{extra} and \textsf{index} options are simple flags. So we
% declare a new conditional for each and, if |\usepackage| includes an option,
% toggle the conditional's state.
%    \begin{macrocode}
\newif\ifemo@extra\emo@extrafalse
\DeclareOption{extra}{\emo@extratrue}
\newif\ifemo@indexing\emo@indexingfalse
\DeclareOption{index}{\emo@indexingtrue}
\ProcessOptions\relax
%    \end{macrocode}
% \end{macro}
% \end{macro}
%
%
% ^^A ----------------------------------------------------------------------------------
% \subsection{Setup Including Dependencies}
%
% The dependency on |inputenc| effectively declares this file's encoding to be
% UTF-8. The XeTeX and LuaTeX engines already expect files to be encoded that
% way and hence ignore the declaration. However, pdfTeX supports other (legacy)
% encodings and needs to be told.
%    \begin{macrocode}
\RequirePackage[utf8]{inputenc}
%    \end{macrocode}
%
% \begin{macro}{\emo@use@unicode}
% \begin{macro}{\emo@use@font}
% \begin{macro}{\emo@use@pdf}
% \begin{macro}{\emo@backend}
% Emo currently supports three different backends for actually rendering emoji,
% namely |\emo@use@unicode| emits Unicode codepoints in a group, |\emo@use@font|
% emits font selection and Unicode codepoints in a group, and |\emo@use@pdf|
% emits PDF graphics. |\emo@backend| represents the currently active backend. To
% select the backend, we interrogate the runtime environment.
% \changes{0.3}{}{Generalize backend selection}
%    \begin{macrocode}
\def\emo@use@unicode{backend:unicode}
\def\emo@use@font{backend:font+unicode}
\def\emo@use@pdf{backend:pdf}
\RequirePackage{iftex}
\ifdefined\HCode
    \let\emo@backend=\emo@use@unicode
\else
\ifluatex
    \let\emo@backend=\emo@use@font
\else
    \let\emo@backend=\emo@use@pdf
\fi
\fi
%    \end{macrocode}
% \end{macro}
% \end{macro}
% \end{macro}
% \end{macro}
%
% With the backend selected, we now require packages used by backend code,
% namely |fontspec| for the |\emo@use@font| backend and |graphicx| for the
% |\emo@use@pdf| backend. The |\emo@use@unicode| backend has no such
% dependencies.
%    \begin{macrocode}
\ifx\emo@backend\emo@use@font
    \RequirePackage{fontspec}
\fi
\ifx\emo@backend\emo@use@pdf
    \RequirePackage{graphicx}
\fi
%    \end{macrocode}
%
% Next, emo requires |xcolor| for formatting highly visible error messages
% within the text. Always including another package that is only used when there
% are errors is not ideal. But when I tried calling |\RequirePackage| for
% |xcolor| from inside the error macro, it didn't work. Alternatively, I could
% make in-text errors optional.
%    \begin{macrocode}
\RequirePackage{xcolor}
%    \end{macrocode}
%
% Finally, emo's options also have dependencies, with \textsf{extra} requiring
% the |xspace| package and \textsf{index} requiring the |index| package:
%    \begin{macrocode}
\ifemo@extra
    \RequirePackage{xspace}
\fi
\ifemo@indexing
    \RequirePackage{index}
\fi
%    \end{macrocode}
%
%
% ^^A ----------------------------------------------------------------------------------
% \subsection{The Emoji Table}
%
% \begin{macro}{\emo@emoji@<name>}
% For each emoji with a PDF graphic in the |emo-graphics| directory and the two
% extras, a macro named |\emo@emoji@|\meta{emoji-name} expands to its Unicode
% sequence. With over 3,000 distinct emoji in Unicode 15, emo relies on a Python
% script for populating the graphics directory and writing the table to the
% \texttt{\jobname.def} file. Since the package code does not change after
% installation but the emoji table may very well change, they are kept separate
% for now. Alternatively, we could use DocStrip to assemble the package file
% from three parts, the code from the previous sections, then the contents of
% the emoji table in |emo.def|, and then all subsequent code.
%    \begin{macrocode}
\ProvidesFile{emo.def}[2023-05-07 v1.0 emo•ji table]


\EmojiBeginGroup{smileys-and-emotion}
\EmojiBeginSubgroup{smileys-and-emotion}{face-smiling}
\DefineEmoji{grinning-face}{😀}
\DefineEmoji{grinning-face-with-big-eyes}{😃}
\DefineEmoji{grinning-face-with-smiling-eyes}{😄}
\DefineEmoji{beaming-face-with-smiling-eyes}{😁}
\DefineEmoji{grinning-squinting-face}{😆}
\DefineEmoji{grinning-face-with-sweat}{😅}
\DefineEmoji{rofl}{🤣}
\DefineEmoji{face-with-tears-of-joy}{😂}
\DefineEmoji{slightly-smiling-face}{🙂}
\DefineEmoji{upside-down-face}{🙃}
\DefineEmoji{melting-face}{🫠}
\DefineEmoji{winking-face}{😉}
\DefineEmoji{smiling-face-with-smiling-eyes}{😊}
\DefineEmoji{smiling-face-with-halo}{😇}
\EmojiEndSubgroup{smileys-and-emotion}{face-smiling}

\EmojiBeginSubgroup{smileys-and-emotion}{face-affection}
\DefineEmoji{smiling-face-with-hearts}{🥰}
\DefineEmoji{smiling-face-with-heart-eyes}{😍}
\DefineEmoji{star-struck}{🤩}
\DefineEmoji{face-blowing-a-kiss}{😘}
\DefineEmoji{kissing-face}{😗}
\DefineEmoji{smiling-face}{☺️}
\DefineEmoji{kissing-face-with-closed-eyes}{😚}
\DefineEmoji{kissing-face-with-smiling-eyes}{😙}
\DefineEmoji{smiling-face-with-tear}{🥲}
\EmojiEndSubgroup{smileys-and-emotion}{face-affection}

\EmojiBeginSubgroup{smileys-and-emotion}{face-tongue}
\DefineEmoji{face-savoring-food}{😋}
\DefineEmoji{face-with-tongue}{😛}
\DefineEmoji{winking-face-with-tongue}{😜}
\DefineEmoji{zany-face}{🤪}
\DefineEmoji{squinting-face-with-tongue}{😝}
\DefineEmoji{money-mouth-face}{🤑}
\EmojiEndSubgroup{smileys-and-emotion}{face-tongue}

\EmojiBeginSubgroup{smileys-and-emotion}{face-hand}
\DefineEmoji{smiling-face-with-open-hands}{🤗}
\DefineEmoji{face-with-hand-over-mouth}{🤭}
\DefineEmoji{face-with-open-eyes-and-hand-over-mouth}{🫢}
\DefineEmoji{face-with-peeking-eye}{🫣}
\DefineEmoji{shushing-face}{🤫}
\DefineEmoji{thinking-face}{🤔}
\DefineEmoji{saluting-face}{🫡}
\EmojiEndSubgroup{smileys-and-emotion}{face-hand}

\EmojiBeginSubgroup{smileys-and-emotion}{face-neutral-skeptical}
\DefineEmoji{zipper-mouth-face}{🤐}
\DefineEmoji{face-with-raised-eyebrow}{🤨}
\DefineEmoji{neutral-face}{😐}
\DefineEmoji{expressionless-face}{😑}
\DefineEmoji{face-without-mouth}{😶}
\DefineEmoji{dotted-line-face}{🫥}
\DefineEmoji{face-in-clouds}{😶‍🌫️}
\DefineEmoji{smirking-face}{😏}
\DefineEmoji{unamused-face}{😒}
\DefineEmoji{face-with-rolling-eyes}{🙄}
\DefineEmoji{grimacing-face}{😬}
\DefineEmoji{face-exhaling}{😮‍💨}
\DefineEmoji{lying-face}{🤥}
\DefineEmoji{shaking-face}{🫨}
\EmojiEndSubgroup{smileys-and-emotion}{face-neutral-skeptical}

\EmojiBeginSubgroup{smileys-and-emotion}{face-sleepy}
\DefineEmoji{relieved-face}{😌}
\DefineEmoji{pensive-face}{😔}
\DefineEmoji{sleepy-face}{😪}
\DefineEmoji{drooling-face}{🤤}
\DefineEmoji{sleeping-face}{😴}
\EmojiEndSubgroup{smileys-and-emotion}{face-sleepy}

\EmojiBeginSubgroup{smileys-and-emotion}{face-unwell}
\DefineEmoji{face-with-medical-mask}{😷}
\DefineEmoji{face-with-thermometer}{🤒}
\DefineEmoji{face-with-head-bandage}{🤕}
\DefineEmoji{nauseated-face}{🤢}
\DefineEmoji{face-vomiting}{🤮}
\DefineEmoji{sneezing-face}{🤧}
\DefineEmoji{hot-face}{🥵}
\DefineEmoji{cold-face}{🥶}
\DefineEmoji{woozy-face}{🥴}
\DefineEmoji{face-with-crossed-out-eyes}{😵}
\DefineEmoji{face-with-spiral-eyes}{😵‍💫}
\DefineEmoji{exploding-head}{🤯}
\EmojiEndSubgroup{smileys-and-emotion}{face-unwell}

\EmojiBeginSubgroup{smileys-and-emotion}{face-hat}
\DefineEmoji{cowboy-hat-face}{🤠}
\DefineEmoji{partying-face}{🥳}
\DefineEmoji{disguised-face}{🥸}
\EmojiEndSubgroup{smileys-and-emotion}{face-hat}

\EmojiBeginSubgroup{smileys-and-emotion}{face-glasses}
\DefineEmoji{smiling-face-with-sunglasses}{😎}
\DefineEmoji{nerd-face}{🤓}
\DefineEmoji{face-with-monocle}{🧐}
\EmojiEndSubgroup{smileys-and-emotion}{face-glasses}

\EmojiBeginSubgroup{smileys-and-emotion}{face-concerned}
\DefineEmoji{confused-face}{😕}
\DefineEmoji{face-with-diagonal-mouth}{🫤}
\DefineEmoji{worried-face}{😟}
\DefineEmoji{slightly-frowning-face}{🙁}
\DefineEmoji{frowning-face}{☹️}
\DefineEmoji{face-with-open-mouth}{😮}
\DefineEmoji{hushed-face}{😯}
\DefineEmoji{astonished-face}{😲}
\DefineEmoji{flushed-face}{😳}
\DefineEmoji{pleading-face}{🥺}
\DefineEmoji{face-holding-back-tears}{🥹}
\DefineEmoji{frowning-face-with-open-mouth}{😦}
\DefineEmoji{anguished-face}{😧}
\DefineEmoji{fearful-face}{😨}
\DefineEmoji{anxious-face-with-sweat}{😰}
\DefineEmoji{sad-but-relieved-face}{😥}
\DefineEmoji{crying-face}{😢}
\DefineEmoji{loudly-crying-face}{😭}
\DefineEmoji{face-screaming-in-fear}{😱}
\DefineEmoji{confounded-face}{😖}
\DefineEmoji{persevering-face}{😣}
\DefineEmoji{disappointed-face}{😞}
\DefineEmoji{downcast-face-with-sweat}{😓}
\DefineEmoji{weary-face}{😩}
\DefineEmoji{tired-face}{😫}
\DefineEmoji{yawning-face}{🥱}
\EmojiEndSubgroup{smileys-and-emotion}{face-concerned}

\EmojiBeginSubgroup{smileys-and-emotion}{face-negative}
\DefineEmoji{face-with-steam-from-nose}{😤}
\DefineEmoji{enraged-face}{😡}
\DefineEmoji{angry-face}{😠}
\DefineEmoji{face-with-symbols-on-mouth}{🤬}
\DefineEmoji{smiling-face-with-horns}{😈}
\DefineEmoji{angry-face-with-horns}{👿}
\DefineEmoji{skull}{💀}
\DefineEmoji{skull-and-crossbones}{☠️}
\EmojiEndSubgroup{smileys-and-emotion}{face-negative}

\EmojiBeginSubgroup{smileys-and-emotion}{face-costume}
\DefineEmoji{pile-of-poo}{💩}
\DefineEmoji{clown-face}{🤡}
\DefineEmoji{ogre}{👹}
\DefineEmoji{goblin}{👺}
\DefineEmoji{ghost}{👻}
\DefineEmoji{alien}{👽}
\DefineEmoji{alien-monster}{👾}
\DefineEmoji{robot}{🤖}
\EmojiEndSubgroup{smileys-and-emotion}{face-costume}

\EmojiBeginSubgroup{smileys-and-emotion}{cat-face}
\DefineEmoji{grinning-cat}{😺}
\DefineEmoji{grinning-cat-with-smiling-eyes}{😸}
\DefineEmoji{cat-with-tears-of-joy}{😹}
\DefineEmoji{smiling-cat-with-heart-eyes}{😻}
\DefineEmoji{cat-with-wry-smile}{😼}
\DefineEmoji{kissing-cat}{😽}
\DefineEmoji{weary-cat}{🙀}
\DefineEmoji{crying-cat}{😿}
\DefineEmoji{pouting-cat}{😾}
\EmojiEndSubgroup{smileys-and-emotion}{cat-face}

\EmojiBeginSubgroup{smileys-and-emotion}{monkey-face}
\DefineEmoji{see-no-evil}{🙈}
\DefineEmoji{hear-no-evil}{🙉}
\DefineEmoji{speak-no-evil}{🙊}
\EmojiEndSubgroup{smileys-and-emotion}{monkey-face}

\EmojiBeginSubgroup{smileys-and-emotion}{heart}
\DefineEmoji{love-letter}{💌}
\DefineEmoji{heart-with-arrow}{💘}
\DefineEmoji{heart-with-ribbon}{💝}
\DefineEmoji{sparkling-heart}{💖}
\DefineEmoji{growing-heart}{💗}
\DefineEmoji{beating-heart}{💓}
\DefineEmoji{revolving-hearts}{💞}
\DefineEmoji{two-hearts}{💕}
\DefineEmoji{heart-decoration}{💟}
\DefineEmoji{heart-exclamation}{❣️}
\DefineEmoji{broken-heart}{💔}
\DefineEmoji{heart-on-fire}{❤️‍🔥}
\DefineEmoji{mending-heart}{❤️‍🩹}
\DefineEmoji{red-heart}{❤️}
\DefineEmoji{pink-heart}{🩷}
\DefineEmoji{orange-heart}{🧡}
\DefineEmoji{yellow-heart}{💛}
\DefineEmoji{green-heart}{💚}
\DefineEmoji{blue-heart}{💙}
\DefineEmoji{light-blue-heart}{🩵}
\DefineEmoji{purple-heart}{💜}
\DefineEmoji{brown-heart}{🤎}
\DefineEmoji{black-heart}{🖤}
\DefineEmoji{grey-heart}{🩶}
\DefineEmoji{white-heart}{🤍}
\EmojiEndSubgroup{smileys-and-emotion}{heart}

\EmojiBeginSubgroup{smileys-and-emotion}{emotion}
\DefineEmoji{kiss-mark}{💋}
\DefineEmoji{hundred-points}{💯}
\DefineEmoji{anger-symbol}{💢}
\DefineEmoji{collision}{💥}
\DefineEmoji{dizzy}{💫}
\DefineEmoji{sweat-droplets}{💦}
\DefineEmoji{dashing-away}{💨}
\DefineEmoji{hole}{🕳️}
\DefineEmoji{speech-balloon}{💬}
\DefineEmoji{eye-in-speech-bubble}{👁️‍🗨️}
\DefineEmoji{left-speech-bubble}{🗨️}
\DefineEmoji{right-anger-bubble}{🗯️}
\DefineEmoji{thought-balloon}{💭}
\DefineEmoji{zzz}{💤}
\EmojiEndSubgroup{smileys-and-emotion}{emotion}
\EmojiEndGroup{smileys-and-emotion}


\EmojiBeginGroup{people-and-body}
\EmojiBeginSubgroup{people-and-body}{hand-fingers-open}
\DefineEmoji{waving-hand}{👋}
\DefineEmoji{raised-back-of-hand}{🤚}
\DefineEmoji{hand-with-fingers-splayed}{🖐️}
\DefineEmoji{raised-hand}{✋}
\DefineEmoji{vulcan-salute}{🖖}
\DefineEmoji{rightwards-hand}{🫱}
\DefineEmoji{leftwards-hand}{🫲}
\DefineEmoji{palm-down}{🫳}
\DefineEmoji{palm-up}{🫴}
\DefineEmoji{leftwards-pushing-hand}{🫷}
\DefineEmoji{rightwards-pushing-hand}{🫸}
\EmojiEndSubgroup{people-and-body}{hand-fingers-open}

\EmojiBeginSubgroup{people-and-body}{hand-fingers-partial}
\DefineEmoji{ok-hand}{👌}
\DefineEmoji{pinched-fingers}{🤌}
\DefineEmoji{pinching-hand}{🤏}
\DefineEmoji{victory-hand}{✌️}
\DefineEmoji{crossed-fingers}{🤞}
\DefineEmoji{hand-with-index-finger-and-thumb-crossed}{🫰}
\DefineEmoji{love-you-gesture}{🤟}
\DefineEmoji{sign-of-the-horns}{🤘}
\DefineEmoji{call-me-hand}{🤙}
\EmojiEndSubgroup{people-and-body}{hand-fingers-partial}

\EmojiBeginSubgroup{people-and-body}{hand-single-finger}
\DefineEmoji{backhand-index-pointing-left}{👈}
\DefineEmoji{backhand-index-pointing-right}{👉}
\DefineEmoji{backhand-index-pointing-up}{👆}
\DefineEmoji{middle-finger}{🖕}
\DefineEmoji{backhand-index-pointing-down}{👇}
\DefineEmoji{index-pointing-up}{☝️}
\DefineEmoji{index-pointing-at-viewer}{🫵}
\EmojiEndSubgroup{people-and-body}{hand-single-finger}

\EmojiBeginSubgroup{people-and-body}{hand-fingers-closed}
\DefineEmoji{thumbs-up}{👍}
\DefineEmoji{thumbs-down}{👎}
\DefineEmoji{raised-fist}{✊}
\DefineEmoji{oncoming-fist}{👊}
\DefineEmoji{left-facing-fist}{🤛}
\DefineEmoji{right-facing-fist}{🤜}
\EmojiEndSubgroup{people-and-body}{hand-fingers-closed}

\EmojiBeginSubgroup{people-and-body}{hands}
\DefineEmoji{clapping-hands}{👏}
\DefineEmoji{raising-hands}{🙌}
\DefineEmoji{heart-hands}{🫶}
\DefineEmoji{open-hands}{👐}
\DefineEmoji{palms-up-together}{🤲}
\DefineEmoji{handshake}{🤝}
\DefineEmoji{folded-hands}{🙏}
\EmojiEndSubgroup{people-and-body}{hands}

\EmojiBeginSubgroup{people-and-body}{hand-prop}
\DefineEmoji{writing-hand}{✍️}
\DefineEmoji{nail-polish}{💅}
\DefineEmoji{selfie}{🤳}
\EmojiEndSubgroup{people-and-body}{hand-prop}

\EmojiBeginSubgroup{people-and-body}{body-parts}
\DefineEmoji{flexed-biceps}{💪}
\DefineEmoji{mechanical-arm}{🦾}
\DefineEmoji{mechanical-leg}{🦿}
\DefineEmoji{leg}{🦵}
\DefineEmoji{foot}{🦶}
\DefineEmoji{ear}{👂}
\DefineEmoji{ear-with-hearing-aid}{🦻}
\DefineEmoji{nose}{👃}
\DefineEmoji{brain}{🧠}
\DefineEmoji{anatomical-heart}{🫀}
\DefineEmoji{lungs}{🫁}
\DefineEmoji{tooth}{🦷}
\DefineEmoji{bone}{🦴}
\DefineEmoji{eyes}{👀}
\DefineEmoji{eye}{👁️}
\DefineEmoji{tongue}{👅}
\DefineEmoji{mouth}{👄}
\DefineEmoji{biting-lip}{🫦}
\EmojiEndSubgroup{people-and-body}{body-parts}

\EmojiBeginSubgroup{people-and-body}{person}
\DefineEmoji{baby}{👶}
\DefineEmoji{child}{🧒}
\DefineEmoji{boy}{👦}
\DefineEmoji{girl}{👧}
\DefineEmoji{person}{🧑}
\DefineEmoji{person-blond-hair}{👱}
\DefineEmoji{person-beard}{🧔}
\DefineEmoji{older-person}{🧓}
\EmojiEndSubgroup{people-and-body}{person}

\EmojiBeginSubgroup{people-and-body}{person-gesture}
\DefineEmoji{person-frowning}{🙍}
\DefineEmoji{person-pouting}{🙎}
\DefineEmoji{person-gesturing-no}{🙅}
\DefineEmoji{person-gesturing-ok}{🙆}
\DefineEmoji{person-tipping-hand}{💁}
\DefineEmoji{person-raising-hand}{🙋}
\DefineEmoji{deaf-person}{🧏}
\DefineEmoji{person-bowing}{🙇}
\DefineEmoji{person-facepalming}{🤦}
\DefineEmoji{person-shrugging}{🤷}
\EmojiEndSubgroup{people-and-body}{person-gesture}

\EmojiBeginSubgroup{people-and-body}{person-role}
\DefineEmoji{health-worker}{🧑‍⚕️}
\DefineEmoji{student}{🧑‍🎓}
\DefineEmoji{teacher}{🧑‍🏫}
\DefineEmoji{judge}{🧑‍⚖️}
\DefineEmoji{farmer}{🧑‍🌾}
\DefineEmoji{cook}{🧑‍🍳}
\DefineEmoji{mechanic}{🧑‍🔧}
\DefineEmoji{factory-worker}{🧑‍🏭}
\DefineEmoji{office-worker}{🧑‍💼}
\DefineEmoji{scientist}{🧑‍🔬}
\DefineEmoji{technologist}{🧑‍💻}
\DefineEmoji{singer}{🧑‍🎤}
\DefineEmoji{artist}{🧑‍🎨}
\DefineEmoji{pilot}{🧑‍✈️}
\DefineEmoji{astronaut}{🧑‍🚀}
\DefineEmoji{firefighter}{🧑‍🚒}
\DefineEmoji{police-officer}{👮}
\DefineEmoji{detective}{🕵️}
\DefineEmoji{guard}{💂}
\DefineEmoji{ninja}{🥷}
\DefineEmoji{construction-worker}{👷}
\DefineEmoji{person-with-crown}{🫅}
\DefineEmoji{prince}{🤴}
\DefineEmoji{princess}{👸}
\DefineEmoji{person-wearing-turban}{👳}
\DefineEmoji{person-with-skullcap}{👲}
\DefineEmoji{person-in-tuxedo}{🤵}
\DefineEmoji{person-with-veil}{👰}
\DefineEmoji{pregnant-person}{🫄}
\DefineEmoji{breast-feeding}{🤱}
\DefineEmoji{person-feeding-baby}{🧑‍🍼}
\EmojiEndSubgroup{people-and-body}{person-role}

\EmojiBeginSubgroup{people-and-body}{person-fantasy}
\DefineEmoji{baby-angel}{👼}
\DefineEmoji{santa-claus}{🎅}
\DefineEmoji{mrs-claus}{🤶}
\DefineEmoji{mx-claus}{🧑‍🎄}
\DefineEmoji{superhero}{🦸}
\DefineEmoji{supervillain}{🦹}
\DefineEmoji{mage}{🧙}
\DefineEmoji{fairy}{🧚}
\DefineEmoji{vampire}{🧛}
\DefineEmoji{merperson}{🧜}
\DefineEmoji{merman}{🧜‍♂️}
\DefineEmoji{mermaid}{🧜‍♀️}
\DefineEmoji{elf}{🧝}
\DefineEmoji{genie}{🧞}
\DefineEmoji{zombie}{🧟}
\DefineEmoji{troll}{🧌}
\EmojiEndSubgroup{people-and-body}{person-fantasy}

\EmojiBeginSubgroup{people-and-body}{person-activity}
\DefineEmoji{person-getting-massage}{💆}
\DefineEmoji{person-getting-haircut}{💇}
\DefineEmoji{person-walking}{🚶}
\DefineEmoji{person-standing}{🧍}
\DefineEmoji{person-kneeling}{🧎}
\DefineEmoji{person-with-white-cane}{🧑‍🦯}
\DefineEmoji{person-in-motorized-wheelchair}{🧑‍🦼}
\DefineEmoji{person-in-manual-wheelchair}{🧑‍🦽}
\DefineEmoji{person-running}{🏃}
\DefineEmoji{person-in-suit-levitating}{🕴️}
\DefineEmoji{people-with-bunny-ears}{👯}
\DefineEmoji{person-in-steamy-room}{🧖}
\DefineEmoji{person-climbing}{🧗}
\EmojiEndSubgroup{people-and-body}{person-activity}

\EmojiBeginSubgroup{people-and-body}{person-sport}
\DefineEmoji{person-fencing}{🤺}
\DefineEmoji{horse-racing}{🏇}
\DefineEmoji{skier}{⛷️}
\DefineEmoji{snowboarder}{🏂}
\DefineEmoji{person-golfing}{🏌️}
\DefineEmoji{person-surfing}{🏄}
\DefineEmoji{person-rowing-boat}{🚣}
\DefineEmoji{person-swimming}{🏊}
\DefineEmoji{person-bouncing-ball}{⛹️}
\DefineEmoji{person-lifting-weights}{🏋️}
\DefineEmoji{person-biking}{🚴}
\DefineEmoji{person-mountain-biking}{🚵}
\DefineEmoji{person-cartwheeling}{🤸}
\DefineEmoji{people-wrestling}{🤼}
\DefineEmoji{person-playing-water-polo}{🤽}
\DefineEmoji{person-playing-handball}{🤾}
\DefineEmoji{person-juggling}{🤹}
\EmojiEndSubgroup{people-and-body}{person-sport}

\EmojiBeginSubgroup{people-and-body}{person-resting}
\DefineEmoji{person-in-lotus-position}{🧘}
\DefineEmoji{person-taking-bath}{🛀}
\DefineEmoji{person-in-bed}{🛌}
\EmojiEndSubgroup{people-and-body}{person-resting}

\EmojiBeginSubgroup{people-and-body}{family}
\DefineEmoji{people-holding-hands}{🧑‍🤝‍🧑}
\DefineEmoji{kiss}{💏}
\DefineEmoji{couple-with-heart}{💑}
\DefineEmoji{family}{👪}
\EmojiEndSubgroup{people-and-body}{family}

\EmojiBeginSubgroup{people-and-body}{person-symbol}
\DefineEmoji{speaking-head}{🗣️}
\DefineEmoji{bust}{👤}
\DefineEmoji{busts}{👥}
\DefineEmoji{people-hugging}{🫂}
\DefineEmoji{footprints}{👣}
\EmojiEndSubgroup{people-and-body}{person-symbol}
\EmojiEndGroup{people-and-body}


\EmojiBeginGroup{animals-and-nature}
\EmojiBeginSubgroup{animals-and-nature}{animal-mammal}
\DefineEmoji{monkey-face}{🐵}
\DefineEmoji{monkey}{🐒}
\DefineEmoji{gorilla}{🦍}
\DefineEmoji{orangutan}{🦧}
\DefineEmoji{dog-face}{🐶}
\DefineEmoji{dog}{🐕}
\DefineEmoji{guide-dog}{🦮}
\DefineEmoji{service-dog}{🐕‍🦺}
\DefineEmoji{poodle}{🐩}
\DefineEmoji{wolf}{🐺}
\DefineEmoji{fox}{🦊}
\DefineEmoji{raccoon}{🦝}
\DefineEmoji{cat-face}{🐱}
\DefineEmoji{cat}{🐈}
\DefineEmoji{black-cat}{🐈‍⬛}
\DefineEmoji{lion}{🦁}
\DefineEmoji{tiger-face}{🐯}
\DefineEmoji{tiger}{🐅}
\DefineEmoji{leopard}{🐆}
\DefineEmoji{horse-face}{🐴}
\DefineEmoji{moose}{🫎}
\DefineEmoji{donkey}{🫏}
\DefineEmoji{horse}{🐎}
\DefineEmoji{unicorn}{🦄}
\DefineEmoji{zebra}{🦓}
\DefineEmoji{deer}{🦌}
\DefineEmoji{bison}{🦬}
\DefineEmoji{cow-face}{🐮}
\DefineEmoji{ox}{🐂}
\DefineEmoji{water-buffalo}{🐃}
\DefineEmoji{cow}{🐄}
\DefineEmoji{pig-face}{🐷}
\DefineEmoji{pig}{🐖}
\DefineEmoji{boar}{🐗}
\DefineEmoji{pig-nose}{🐽}
\DefineEmoji{ram}{🐏}
\DefineEmoji{ewe}{🐑}
\DefineEmoji{goat}{🐐}
\DefineEmoji{camel}{🐪}
\DefineEmoji{two-hump-camel}{🐫}
\DefineEmoji{llama}{🦙}
\DefineEmoji{giraffe}{🦒}
\DefineEmoji{elephant}{🐘}
\DefineEmoji{mammoth}{🦣}
\DefineEmoji{rhinoceros}{🦏}
\DefineEmoji{hippopotamus}{🦛}
\DefineEmoji{mouse-face}{🐭}
\DefineEmoji{mouse}{🐁}
\DefineEmoji{rat}{🐀}
\DefineEmoji{hamster}{🐹}
\DefineEmoji{rabbit-face}{🐰}
\DefineEmoji{rabbit}{🐇}
\DefineEmoji{chipmunk}{🐿️}
\DefineEmoji{beaver}{🦫}
\DefineEmoji{hedgehog}{🦔}
\DefineEmoji{bat}{🦇}
\DefineEmoji{bear}{🐻}
\DefineEmoji{polar-bear}{🐻‍❄️}
\DefineEmoji{koala}{🐨}
\DefineEmoji{panda}{🐼}
\DefineEmoji{sloth}{🦥}
\DefineEmoji{otter}{🦦}
\DefineEmoji{skunk}{🦨}
\DefineEmoji{kangaroo}{🦘}
\DefineEmoji{badger}{🦡}
\DefineEmoji{paw-prints}{🐾}
\EmojiEndSubgroup{animals-and-nature}{animal-mammal}

\EmojiBeginSubgroup{animals-and-nature}{animal-bird}
\DefineEmoji{turkey}{🦃}
\DefineEmoji{chicken}{🐔}
\DefineEmoji{rooster}{🐓}
\DefineEmoji{hatching-chick}{🐣}
\DefineEmoji{baby-chick}{🐤}
\DefineEmoji{front-facing-baby-chick}{🐥}
\DefineEmoji{bird}{🐦}
\DefineEmoji{penguin}{🐧}
\DefineEmoji{dove}{🕊️}
\DefineEmoji{eagle}{🦅}
\DefineEmoji{duck}{🦆}
\DefineEmoji{swan}{🦢}
\DefineEmoji{owl}{🦉}
\DefineEmoji{dodo}{🦤}
\DefineEmoji{feather}{🪶}
\DefineEmoji{flamingo}{🦩}
\DefineEmoji{peacock}{🦚}
\DefineEmoji{parrot}{🦜}
\DefineEmoji{wing}{🪽}
\DefineEmoji{black-bird}{🐦‍⬛}
\DefineEmoji{goose}{🪿}
\EmojiEndSubgroup{animals-and-nature}{animal-bird}

\EmojiBeginSubgroup{animals-and-nature}{animal-amphibian}
\DefineEmoji{frog}{🐸}
\EmojiEndSubgroup{animals-and-nature}{animal-amphibian}

\EmojiBeginSubgroup{animals-and-nature}{animal-reptile}
\DefineEmoji{crocodile}{🐊}
\DefineEmoji{turtle}{🐢}
\DefineEmoji{lizard}{🦎}
\DefineEmoji{snake}{🐍}
\DefineEmoji{dragon-face}{🐲}
\DefineEmoji{dragon}{🐉}
\DefineEmoji{sauropod}{🦕}
\DefineEmoji{t-rex}{🦖}
\EmojiEndSubgroup{animals-and-nature}{animal-reptile}

\EmojiBeginSubgroup{animals-and-nature}{animal-marine}
\DefineEmoji{spouting-whale}{🐳}
\DefineEmoji{whale}{🐋}
\DefineEmoji{dolphin}{🐬}
\DefineEmoji{seal}{🦭}
\DefineEmoji{fish}{🐟}
\DefineEmoji{tropical-fish}{🐠}
\DefineEmoji{blowfish}{🐡}
\DefineEmoji{shark}{🦈}
\DefineEmoji{octopus}{🐙}
\DefineEmoji{spiral-shell}{🐚}
\DefineEmoji{coral}{🪸}
\DefineEmoji{jellyfish}{🪼}
\EmojiEndSubgroup{animals-and-nature}{animal-marine}

\EmojiBeginSubgroup{animals-and-nature}{animal-bug}
\DefineEmoji{snail}{🐌}
\DefineEmoji{butterfly}{🦋}
\DefineEmoji{bug}{🐛}
\DefineEmoji{ant}{🐜}
\DefineEmoji{honeybee}{🐝}
\DefineEmoji{beetle}{🪲}
\DefineEmoji{lady-beetle}{🐞}
\DefineEmoji{cricket}{🦗}
\DefineEmoji{cockroach}{🪳}
\DefineEmoji{spider}{🕷️}
\DefineEmoji{spider-web}{🕸️}
\DefineEmoji{scorpion}{🦂}
\DefineEmoji{mosquito}{🦟}
\DefineEmoji{fly}{🪰}
\DefineEmoji{worm}{🪱}
\DefineEmoji{microbe}{🦠}
\EmojiEndSubgroup{animals-and-nature}{animal-bug}

\EmojiBeginSubgroup{animals-and-nature}{plant-flower}
\DefineEmoji{bouquet}{💐}
\DefineEmoji{cherry-blossom}{🌸}
\DefineEmoji{white-flower}{💮}
\DefineEmoji{lotus}{🪷}
\DefineEmoji{rosette}{🏵️}
\DefineEmoji{rose}{🌹}
\DefineEmoji{wilted-flower}{🥀}
\DefineEmoji{hibiscus}{🌺}
\DefineEmoji{sunflower}{🌻}
\DefineEmoji{blossom}{🌼}
\DefineEmoji{tulip}{🌷}
\DefineEmoji{hyacinth}{🪻}
\EmojiEndSubgroup{animals-and-nature}{plant-flower}

\EmojiBeginSubgroup{animals-and-nature}{plant-other}
\DefineEmoji{seedling}{🌱}
\DefineEmoji{potted-plant}{🪴}
\DefineEmoji{evergreen-tree}{🌲}
\DefineEmoji{deciduous-tree}{🌳}
\DefineEmoji{palm-tree}{🌴}
\DefineEmoji{cactus}{🌵}
\DefineEmoji{sheaf-of-rice}{🌾}
\DefineEmoji{herb}{🌿}
\DefineEmoji{shamrock}{☘️}
\DefineEmoji{four-leaf-clover}{🍀}
\DefineEmoji{maple-leaf}{🍁}
\DefineEmoji{fallen-leaf}{🍂}
\DefineEmoji{leaf-fluttering-in-wind}{🍃}
\DefineEmoji{empty-nest}{🪹}
\DefineEmoji{nest-with-eggs}{🪺}
\DefineEmoji{mushroom}{🍄}
\EmojiEndSubgroup{animals-and-nature}{plant-other}
\EmojiEndGroup{animals-and-nature}


\EmojiBeginGroup{food-and-drink}
\EmojiBeginSubgroup{food-and-drink}{food-fruit}
\DefineEmoji{grapes}{🍇}
\DefineEmoji{melon}{🍈}
\DefineEmoji{watermelon}{🍉}
\DefineEmoji{tangerine}{🍊}
\DefineEmoji{lemon}{🍋}
\DefineEmoji{banana}{🍌}
\DefineEmoji{pineapple}{🍍}
\DefineEmoji{mango}{🥭}
\DefineEmoji{red-apple}{🍎}
\DefineEmoji{green-apple}{🍏}
\DefineEmoji{pear}{🍐}
\DefineEmoji{peach}{🍑}
\DefineEmoji{cherries}{🍒}
\DefineEmoji{strawberry}{🍓}
\DefineEmoji{blueberries}{🫐}
\DefineEmoji{kiwi-fruit}{🥝}
\DefineEmoji{tomato}{🍅}
\DefineEmoji{olive}{🫒}
\DefineEmoji{coconut}{🥥}
\EmojiEndSubgroup{food-and-drink}{food-fruit}

\EmojiBeginSubgroup{food-and-drink}{food-vegetable}
\DefineEmoji{avocado}{🥑}
\DefineEmoji{eggplant}{🍆}
\DefineEmoji{potato}{🥔}
\DefineEmoji{carrot}{🥕}
\DefineEmoji{ear-of-corn}{🌽}
\DefineEmoji{hot-pepper}{🌶️}
\DefineEmoji{bell-pepper}{🫑}
\DefineEmoji{cucumber}{🥒}
\DefineEmoji{leafy-green}{🥬}
\DefineEmoji{broccoli}{🥦}
\DefineEmoji{garlic}{🧄}
\DefineEmoji{onion}{🧅}
\DefineEmoji{peanuts}{🥜}
\DefineEmoji{beans}{🫘}
\DefineEmoji{chestnut}{🌰}
\DefineEmoji{ginger-root}{🫚}
\DefineEmoji{pea-pod}{🫛}
\EmojiEndSubgroup{food-and-drink}{food-vegetable}

\EmojiBeginSubgroup{food-and-drink}{food-prepared}
\DefineEmoji{bread}{🍞}
\DefineEmoji{croissant}{🥐}
\DefineEmoji{baguette-bread}{🥖}
\DefineEmoji{flatbread}{🫓}
\DefineEmoji{pretzel}{🥨}
\DefineEmoji{bagel}{🥯}
\DefineEmoji{pancakes}{🥞}
\DefineEmoji{waffle}{🧇}
\DefineEmoji{cheese-wedge}{🧀}
\DefineEmoji{meat-on-bone}{🍖}
\DefineEmoji{poultry-leg}{🍗}
\DefineEmoji{cut-of-meat}{🥩}
\DefineEmoji{bacon}{🥓}
\DefineEmoji{hamburger}{🍔}
\DefineEmoji{french-fries}{🍟}
\DefineEmoji{pizza}{🍕}
\DefineEmoji{hot-dog}{🌭}
\DefineEmoji{sandwich}{🥪}
\DefineEmoji{taco}{🌮}
\DefineEmoji{burrito}{🌯}
\DefineEmoji{tamale}{🫔}
\DefineEmoji{stuffed-flatbread}{🥙}
\DefineEmoji{falafel}{🧆}
\DefineEmoji{egg}{🥚}
\DefineEmoji{cooking}{🍳}
\DefineEmoji{shallow-pan-of-food}{🥘}
\DefineEmoji{pot-of-food}{🍲}
\DefineEmoji{fondue}{🫕}
\DefineEmoji{bowl-with-spoon}{🥣}
\DefineEmoji{green-salad}{🥗}
\DefineEmoji{popcorn}{🍿}
\DefineEmoji{butter}{🧈}
\DefineEmoji{salt}{🧂}
\DefineEmoji{canned-food}{🥫}
\EmojiEndSubgroup{food-and-drink}{food-prepared}

\EmojiBeginSubgroup{food-and-drink}{food-asian}
\DefineEmoji{bento-box}{🍱}
\DefineEmoji{rice-cracker}{🍘}
\DefineEmoji{rice-ball}{🍙}
\DefineEmoji{cooked-rice}{🍚}
\DefineEmoji{curry-rice}{🍛}
\DefineEmoji{steaming-bowl}{🍜}
\DefineEmoji{spaghetti}{🍝}
\DefineEmoji{roasted-sweet-potato}{🍠}
\DefineEmoji{oden}{🍢}
\DefineEmoji{sushi}{🍣}
\DefineEmoji{fried-shrimp}{🍤}
\DefineEmoji{fish-cake-with-swirl}{🍥}
\DefineEmoji{moon-cake}{🥮}
\DefineEmoji{dango}{🍡}
\DefineEmoji{dumpling}{🥟}
\DefineEmoji{fortune-cookie}{🥠}
\DefineEmoji{takeout-box}{🥡}
\EmojiEndSubgroup{food-and-drink}{food-asian}

\EmojiBeginSubgroup{food-and-drink}{food-marine}
\DefineEmoji{crab}{🦀}
\DefineEmoji{lobster}{🦞}
\DefineEmoji{shrimp}{🦐}
\DefineEmoji{squid}{🦑}
\DefineEmoji{oyster}{🦪}
\EmojiEndSubgroup{food-and-drink}{food-marine}

\EmojiBeginSubgroup{food-and-drink}{food-sweet}
\DefineEmoji{soft-ice-cream}{🍦}
\DefineEmoji{shaved-ice}{🍧}
\DefineEmoji{ice-cream}{🍨}
\DefineEmoji{doughnut}{🍩}
\DefineEmoji{cookie}{🍪}
\DefineEmoji{birthday-cake}{🎂}
\DefineEmoji{shortcake}{🍰}
\DefineEmoji{cupcake}{🧁}
\DefineEmoji{pie}{🥧}
\DefineEmoji{chocolate-bar}{🍫}
\DefineEmoji{candy}{🍬}
\DefineEmoji{lollipop}{🍭}
\DefineEmoji{custard}{🍮}
\DefineEmoji{honey-pot}{🍯}
\EmojiEndSubgroup{food-and-drink}{food-sweet}

\EmojiBeginSubgroup{food-and-drink}{drink}
\DefineEmoji{baby-bottle}{🍼}
\DefineEmoji{glass-of-milk}{🥛}
\DefineEmoji{hot-beverage}{☕}
\DefineEmoji{teapot}{🫖}
\DefineEmoji{teacup-without-handle}{🍵}
\DefineEmoji{sake}{🍶}
\DefineEmoji{bottle-with-popping-cork}{🍾}
\DefineEmoji{wine-glass}{🍷}
\DefineEmoji{cocktail-glass}{🍸}
\DefineEmoji{tropical-drink}{🍹}
\DefineEmoji{beer-mug}{🍺}
\DefineEmoji{clinking-beer-mugs}{🍻}
\DefineEmoji{clinking-glasses}{🥂}
\DefineEmoji{tumbler-glass}{🥃}
\DefineEmoji{pouring-liquid}{🫗}
\DefineEmoji{cup-with-straw}{🥤}
\DefineEmoji{bubble-tea}{🧋}
\DefineEmoji{beverage-box}{🧃}
\DefineEmoji{mate}{🧉}
\DefineEmoji{ice}{🧊}
\EmojiEndSubgroup{food-and-drink}{drink}

\EmojiBeginSubgroup{food-and-drink}{dishware}
\DefineEmoji{chopsticks}{🥢}
\DefineEmoji{fork-and-knife-with-plate}{🍽️}
\DefineEmoji{fork-and-knife}{🍴}
\DefineEmoji{spoon}{🥄}
\DefineEmoji{kitchen-knife}{🔪}
\DefineEmoji{jar}{🫙}
\DefineEmoji{amphora}{🏺}
\EmojiEndSubgroup{food-and-drink}{dishware}
\EmojiEndGroup{food-and-drink}


\EmojiBeginGroup{travel-and-places}
\EmojiBeginSubgroup{travel-and-places}{place-map}
\DefineEmoji{globe-africa-europe}{🌍}
\DefineEmoji{globe-americas}{🌎}
\DefineEmoji{globe-asia-australia}{🌏}
\DefineEmoji{globe-with-meridians}{🌐}
\DefineEmoji{world-map}{🗺️}
\DefineEmoji{map-of-japan}{🗾}
\DefineEmoji{compass}{🧭}
\EmojiEndSubgroup{travel-and-places}{place-map}

\EmojiBeginSubgroup{travel-and-places}{place-geographic}
\DefineEmoji{snow-capped-mountain}{🏔️}
\DefineEmoji{mountain}{⛰️}
\DefineEmoji{volcano}{🌋}
\DefineEmoji{mount-fuji}{🗻}
\DefineEmoji{camping}{🏕️}
\DefineEmoji{beach-with-umbrella}{🏖️}
\DefineEmoji{desert}{🏜️}
\DefineEmoji{desert-island}{🏝️}
\DefineEmoji{national-park}{🏞️}
\EmojiEndSubgroup{travel-and-places}{place-geographic}

\EmojiBeginSubgroup{travel-and-places}{place-building}
\DefineEmoji{stadium}{🏟️}
\DefineEmoji{classical-building}{🏛️}
\DefineEmoji{building-construction}{🏗️}
\DefineEmoji{brick}{🧱}
\DefineEmoji{rock}{🪨}
\DefineEmoji{wood}{🪵}
\DefineEmoji{hut}{🛖}
\DefineEmoji{houses}{🏘️}
\DefineEmoji{derelict-house}{🏚️}
\DefineEmoji{house}{🏠}
\DefineEmoji{house-with-garden}{🏡}
\DefineEmoji{office-building}{🏢}
\DefineEmoji{japanese-post-office}{🏣}
\DefineEmoji{post-office}{🏤}
\DefineEmoji{hospital}{🏥}
\DefineEmoji{bank}{🏦}
\DefineEmoji{hotel}{🏨}
\DefineEmoji{love-hotel}{🏩}
\DefineEmoji{convenience-store}{🏪}
\DefineEmoji{school}{🏫}
\DefineEmoji{department-store}{🏬}
\DefineEmoji{factory}{🏭}
\DefineEmoji{japanese-castle}{🏯}
\DefineEmoji{castle}{🏰}
\DefineEmoji{wedding}{💒}
\DefineEmoji{tokyo-tower}{🗼}
\DefineEmoji{statue-of-liberty}{🗽}
\EmojiEndSubgroup{travel-and-places}{place-building}

\EmojiBeginSubgroup{travel-and-places}{place-religious}
\DefineEmoji{church}{⛪}
\DefineEmoji{mosque}{🕌}
\DefineEmoji{hindu-temple}{🛕}
\DefineEmoji{synagogue}{🕍}
\DefineEmoji{shinto-shrine}{⛩️}
\DefineEmoji{kaaba}{🕋}
\EmojiEndSubgroup{travel-and-places}{place-religious}

\EmojiBeginSubgroup{travel-and-places}{place-other}
\DefineEmoji{fountain}{⛲}
\DefineEmoji{tent}{⛺}
\DefineEmoji{foggy}{🌁}
\DefineEmoji{night-with-stars}{🌃}
\DefineEmoji{cityscape}{🏙️}
\DefineEmoji{sunrise-over-mountains}{🌄}
\DefineEmoji{sunrise}{🌅}
\DefineEmoji{cityscape-at-dusk}{🌆}
\DefineEmoji{sunset}{🌇}
\DefineEmoji{bridge-at-night}{🌉}
\DefineEmoji{hot-springs}{♨️}
\DefineEmoji{carousel-horse}{🎠}
\DefineEmoji{playground-slide}{🛝}
\DefineEmoji{ferris-wheel}{🎡}
\DefineEmoji{roller-coaster}{🎢}
\DefineEmoji{barber-pole}{💈}
\DefineEmoji{circus-tent}{🎪}
\EmojiEndSubgroup{travel-and-places}{place-other}

\EmojiBeginSubgroup{travel-and-places}{transport-ground}
\DefineEmoji{locomotive}{🚂}
\DefineEmoji{railway-car}{🚃}
\DefineEmoji{high-speed-train}{🚄}
\DefineEmoji{bullet-train}{🚅}
\DefineEmoji{train}{🚆}
\DefineEmoji{metro}{🚇}
\DefineEmoji{light-rail}{🚈}
\DefineEmoji{station}{🚉}
\DefineEmoji{tram}{🚊}
\DefineEmoji{monorail}{🚝}
\DefineEmoji{mountain-railway}{🚞}
\DefineEmoji{tram-car}{🚋}
\DefineEmoji{bus}{🚌}
\DefineEmoji{oncoming-bus}{🚍}
\DefineEmoji{trolleybus}{🚎}
\DefineEmoji{minibus}{🚐}
\DefineEmoji{ambulance}{🚑}
\DefineEmoji{fire-engine}{🚒}
\DefineEmoji{police-car}{🚓}
\DefineEmoji{oncoming-police-car}{🚔}
\DefineEmoji{taxi}{🚕}
\DefineEmoji{oncoming-taxi}{🚖}
\DefineEmoji{automobile}{🚗}
\DefineEmoji{oncoming-automobile}{🚘}
\DefineEmoji{sport-utility-vehicle}{🚙}
\DefineEmoji{pickup-truck}{🛻}
\DefineEmoji{delivery-truck}{🚚}
\DefineEmoji{articulated-lorry}{🚛}
\DefineEmoji{tractor}{🚜}
\DefineEmoji{racing-car}{🏎️}
\DefineEmoji{motorcycle}{🏍️}
\DefineEmoji{motor-scooter}{🛵}
\DefineEmoji{manual-wheelchair}{🦽}
\DefineEmoji{motorized-wheelchair}{🦼}
\DefineEmoji{auto-rickshaw}{🛺}
\DefineEmoji{bicycle}{🚲}
\DefineEmoji{kick-scooter}{🛴}
\DefineEmoji{skateboard}{🛹}
\DefineEmoji{roller-skate}{🛼}
\DefineEmoji{bus-stop}{🚏}
\DefineEmoji{motorway}{🛣️}
\DefineEmoji{railway-track}{🛤️}
\DefineEmoji{oil-drum}{🛢️}
\DefineEmoji{fuel-pump}{⛽}
\DefineEmoji{wheel}{🛞}
\DefineEmoji{police-car-light}{🚨}
\DefineEmoji{horizontal-traffic-light}{🚥}
\DefineEmoji{vertical-traffic-light}{🚦}
\DefineEmoji{stop-sign}{🛑}
\DefineEmoji{construction}{🚧}
\EmojiEndSubgroup{travel-and-places}{transport-ground}

\EmojiBeginSubgroup{travel-and-places}{transport-water}
\DefineEmoji{anchor}{⚓}
\DefineEmoji{ring-buoy}{🛟}
\DefineEmoji{sailboat}{⛵}
\DefineEmoji{canoe}{🛶}
\DefineEmoji{speedboat}{🚤}
\DefineEmoji{passenger-ship}{🛳️}
\DefineEmoji{ferry}{⛴️}
\DefineEmoji{motor-boat}{🛥️}
\DefineEmoji{ship}{🚢}
\EmojiEndSubgroup{travel-and-places}{transport-water}

\EmojiBeginSubgroup{travel-and-places}{transport-air}
\DefineEmoji{airplane}{✈️}
\DefineEmoji{small-airplane}{🛩️}
\DefineEmoji{airplane-departure}{🛫}
\DefineEmoji{airplane-arrival}{🛬}
\DefineEmoji{parachute}{🪂}
\DefineEmoji{seat}{💺}
\DefineEmoji{helicopter}{🚁}
\DefineEmoji{suspension-railway}{🚟}
\DefineEmoji{mountain-cableway}{🚠}
\DefineEmoji{aerial-tramway}{🚡}
\DefineEmoji{satellite}{🛰️}
\DefineEmoji{rocket}{🚀}
\DefineEmoji{flying-saucer}{🛸}
\EmojiEndSubgroup{travel-and-places}{transport-air}

\EmojiBeginSubgroup{travel-and-places}{hotel}
\DefineEmoji{bellhop-bell}{🛎️}
\DefineEmoji{luggage}{🧳}
\EmojiEndSubgroup{travel-and-places}{hotel}

\EmojiBeginSubgroup{travel-and-places}{time}
\DefineEmoji{hourglass-done}{⌛}
\DefineEmoji{hourglass-not-done}{⏳}
\DefineEmoji{watch}{⌚}
\DefineEmoji{alarm-clock}{⏰}
\DefineEmoji{stopwatch}{⏱️}
\DefineEmoji{timer-clock}{⏲️}
\DefineEmoji{mantelpiece-clock}{🕰️}
\DefineEmoji{twelve-oclock}{🕛}
\DefineEmoji{twelve-thirty}{🕧}
\DefineEmoji{one-oclock}{🕐}
\DefineEmoji{one-thirty}{🕜}
\DefineEmoji{two-oclock}{🕑}
\DefineEmoji{two-thirty}{🕝}
\DefineEmoji{three-oclock}{🕒}
\DefineEmoji{three-thirty}{🕞}
\DefineEmoji{four-oclock}{🕓}
\DefineEmoji{four-thirty}{🕟}
\DefineEmoji{five-oclock}{🕔}
\DefineEmoji{five-thirty}{🕠}
\DefineEmoji{six-oclock}{🕕}
\DefineEmoji{six-thirty}{🕡}
\DefineEmoji{seven-oclock}{🕖}
\DefineEmoji{seven-thirty}{🕢}
\DefineEmoji{eight-oclock}{🕗}
\DefineEmoji{eight-thirty}{🕣}
\DefineEmoji{nine-oclock}{🕘}
\DefineEmoji{nine-thirty}{🕤}
\DefineEmoji{ten-oclock}{🕙}
\DefineEmoji{ten-thirty}{🕥}
\DefineEmoji{eleven-oclock}{🕚}
\DefineEmoji{eleven-thirty}{🕦}
\EmojiEndSubgroup{travel-and-places}{time}

\EmojiBeginSubgroup{travel-and-places}{sky-and-weather}
\DefineEmoji{new-moon}{🌑}
\DefineEmoji{waxing-crescent-moon}{🌒}
\DefineEmoji{first-quarter-moon}{🌓}
\DefineEmoji{waxing-gibbous-moon}{🌔}
\DefineEmoji{full-moon}{🌕}
\DefineEmoji{waning-gibbous-moon}{🌖}
\DefineEmoji{last-quarter-moon}{🌗}
\DefineEmoji{waning-crescent-moon}{🌘}
\DefineEmoji{crescent-moon}{🌙}
\DefineEmoji{new-moon-face}{🌚}
\DefineEmoji{first-quarter-moon-face}{🌛}
\DefineEmoji{last-quarter-moon-face}{🌜}
\DefineEmoji{thermometer}{🌡️}
\DefineEmoji{sun}{☀️}
\DefineEmoji{full-moon-face}{🌝}
\DefineEmoji{sun-with-face}{🌞}
\DefineEmoji{ringed-planet}{🪐}
\DefineEmoji{star}{⭐}
\DefineEmoji{glowing-star}{🌟}
\DefineEmoji{shooting-star}{🌠}
\DefineEmoji{milky-way}{🌌}
\DefineEmoji{cloud}{☁️}
\DefineEmoji{sun-behind-cloud}{⛅}
\DefineEmoji{cloud-with-lightning-and-rain}{⛈️}
\DefineEmoji{sun-behind-small-cloud}{🌤️}
\DefineEmoji{sun-behind-large-cloud}{🌥️}
\DefineEmoji{sun-behind-rain-cloud}{🌦️}
\DefineEmoji{cloud-with-rain}{🌧️}
\DefineEmoji{cloud-with-snow}{🌨️}
\DefineEmoji{cloud-with-lightning}{🌩️}
\DefineEmoji{tornado}{🌪️}
\DefineEmoji{fog}{🌫️}
\DefineEmoji{wind-face}{🌬️}
\DefineEmoji{cyclone}{🌀}
\DefineEmoji{rainbow}{🌈}
\DefineEmoji{closed-umbrella}{🌂}
\DefineEmoji{umbrella}{☂️}
\DefineEmoji{umbrella-with-rain-drops}{☔}
\DefineEmoji{umbrella-on-ground}{⛱️}
\DefineEmoji{high-voltage}{⚡}
\DefineEmoji{snowflake}{❄️}
\DefineEmoji{snowman}{☃️}
\DefineEmoji{snowman-without-snow}{⛄}
\DefineEmoji{comet}{☄️}
\DefineEmoji{fire}{🔥}
\DefineEmoji{droplet}{💧}
\DefineEmoji{water-wave}{🌊}
\EmojiEndSubgroup{travel-and-places}{sky-and-weather}
\EmojiEndGroup{travel-and-places}


\EmojiBeginGroup{activities}
\EmojiBeginSubgroup{activities}{event}
\DefineEmoji{jack-o-lantern}{🎃}
\DefineEmoji{christmas-tree}{🎄}
\DefineEmoji{fireworks}{🎆}
\DefineEmoji{sparkler}{🎇}
\DefineEmoji{firecracker}{🧨}
\DefineEmoji{sparkles}{✨}
\DefineEmoji{balloon}{🎈}
\DefineEmoji{party-popper}{🎉}
\DefineEmoji{confetti-ball}{🎊}
\DefineEmoji{tanabata-tree}{🎋}
\DefineEmoji{pine-decoration}{🎍}
\DefineEmoji{japanese-dolls}{🎎}
\DefineEmoji{carp-streamer}{🎏}
\DefineEmoji{wind-chime}{🎐}
\DefineEmoji{moon-viewing-ceremony}{🎑}
\DefineEmoji{red-envelope}{🧧}
\DefineEmoji{ribbon}{🎀}
\DefineEmoji{wrapped-gift}{🎁}
\DefineEmoji{reminder-ribbon}{🎗️}
\DefineEmoji{admission-tickets}{🎟️}
\DefineEmoji{ticket}{🎫}
\EmojiEndSubgroup{activities}{event}

\EmojiBeginSubgroup{activities}{award-medal}
\DefineEmoji{military-medal}{🎖️}
\DefineEmoji{trophy}{🏆}
\DefineEmoji{sports-medal}{🏅}
\DefineEmoji{1st-place-medal}{🥇}
\DefineEmoji{2nd-place-medal}{🥈}
\DefineEmoji{3rd-place-medal}{🥉}
\EmojiEndSubgroup{activities}{award-medal}

\EmojiBeginSubgroup{activities}{sport}
\DefineEmoji{soccer-ball}{⚽}
\DefineEmoji{baseball}{⚾}
\DefineEmoji{softball}{🥎}
\DefineEmoji{basketball}{🏀}
\DefineEmoji{volleyball}{🏐}
\DefineEmoji{american-football}{🏈}
\DefineEmoji{rugby-football}{🏉}
\DefineEmoji{tennis}{🎾}
\DefineEmoji{flying-disc}{🥏}
\DefineEmoji{bowling}{🎳}
\DefineEmoji{cricket-game}{🏏}
\DefineEmoji{field-hockey}{🏑}
\DefineEmoji{ice-hockey}{🏒}
\DefineEmoji{lacrosse}{🥍}
\DefineEmoji{ping-pong}{🏓}
\DefineEmoji{badminton}{🏸}
\DefineEmoji{boxing-glove}{🥊}
\DefineEmoji{martial-arts-uniform}{🥋}
\DefineEmoji{goal-net}{🥅}
\DefineEmoji{flag-in-hole}{⛳}
\DefineEmoji{ice-skate}{⛸️}
\DefineEmoji{fishing-pole}{🎣}
\DefineEmoji{diving-mask}{🤿}
\DefineEmoji{running-shirt}{🎽}
\DefineEmoji{skis}{🎿}
\DefineEmoji{sled}{🛷}
\DefineEmoji{curling-stone}{🥌}
\EmojiEndSubgroup{activities}{sport}

\EmojiBeginSubgroup{activities}{game}
\DefineEmoji{bullseye}{🎯}
\DefineEmoji{yo-yo}{🪀}
\DefineEmoji{kite}{🪁}
\DefineEmoji{water-pistol}{🔫}
\DefineEmoji{pool-8-ball}{🎱}
\DefineEmoji{crystal-ball}{🔮}
\DefineEmoji{magic-wand}{🪄}
\DefineEmoji{video-game}{🎮}
\DefineEmoji{joystick}{🕹️}
\DefineEmoji{slot-machine}{🎰}
\DefineEmoji{game-die}{🎲}
\DefineEmoji{puzzle-piece}{🧩}
\DefineEmoji{teddy-bear}{🧸}
\DefineEmoji{piñata}{🪅}
\DefineEmoji{mirror-ball}{🪩}
\DefineEmoji{nesting-dolls}{🪆}
\DefineEmoji{spade-suit}{♠️}
\DefineEmoji{heart-suit}{♥️}
\DefineEmoji{diamond-suit}{♦️}
\DefineEmoji{club-suit}{♣️}
\DefineEmoji{chess-pawn}{♟️}
\DefineEmoji{joker}{🃏}
\DefineEmoji{mahjong-red-dragon}{🀄}
\DefineEmoji{flower-playing-cards}{🎴}
\EmojiEndSubgroup{activities}{game}

\EmojiBeginSubgroup{activities}{arts-and-crafts}
\DefineEmoji{performing-arts}{🎭}
\DefineEmoji{framed-picture}{🖼️}
\DefineEmoji{artist-palette}{🎨}
\DefineEmoji{thread}{🧵}
\DefineEmoji{sewing-needle}{🪡}
\DefineEmoji{yarn}{🧶}
\DefineEmoji{knot}{🪢}
\EmojiEndSubgroup{activities}{arts-and-crafts}
\EmojiEndGroup{activities}


\EmojiBeginGroup{objects}
\EmojiBeginSubgroup{objects}{clothing}
\DefineEmoji{glasses}{👓}
\DefineEmoji{sunglasses}{🕶️}
\DefineEmoji{goggles}{🥽}
\DefineEmoji{lab-coat}{🥼}
\DefineEmoji{safety-vest}{🦺}
\DefineEmoji{necktie}{👔}
\DefineEmoji{t-shirt}{👕}
\DefineEmoji{jeans}{👖}
\DefineEmoji{scarf}{🧣}
\DefineEmoji{gloves}{🧤}
\DefineEmoji{coat}{🧥}
\DefineEmoji{socks}{🧦}
\DefineEmoji{dress}{👗}
\DefineEmoji{kimono}{👘}
\DefineEmoji{sari}{🥻}
\DefineEmoji{one-piece-swimsuit}{🩱}
\DefineEmoji{briefs}{🩲}
\DefineEmoji{shorts}{🩳}
\DefineEmoji{bikini}{👙}
\DefineEmoji{womans-clothes}{👚}
\DefineEmoji{folding-hand-fan}{🪭}
\DefineEmoji{purse}{👛}
\DefineEmoji{handbag}{👜}
\DefineEmoji{clutch-bag}{👝}
\DefineEmoji{shopping-bags}{🛍️}
\DefineEmoji{backpack}{🎒}
\DefineEmoji{thong-sandal}{🩴}
\DefineEmoji{mans-shoe}{👞}
\DefineEmoji{running-shoe}{👟}
\DefineEmoji{hiking-boot}{🥾}
\DefineEmoji{flat-shoe}{🥿}
\DefineEmoji{high-heeled-shoe}{👠}
\DefineEmoji{womans-sandal}{👡}
\DefineEmoji{ballet-shoes}{🩰}
\DefineEmoji{womans-boot}{👢}
\DefineEmoji{hair-pick}{🪮}
\DefineEmoji{crown}{👑}
\DefineEmoji{womans-hat}{👒}
\DefineEmoji{top-hat}{🎩}
\DefineEmoji{graduation-cap}{🎓}
\DefineEmoji{billed-cap}{🧢}
\DefineEmoji{military-helmet}{🪖}
\DefineEmoji{rescue-workers-helmet}{⛑️}
\DefineEmoji{prayer-beads}{📿}
\DefineEmoji{lipstick}{💄}
\DefineEmoji{ring}{💍}
\DefineEmoji{gem-stone}{💎}
\EmojiEndSubgroup{objects}{clothing}

\EmojiBeginSubgroup{objects}{sound}
\DefineEmoji{muted-speaker}{🔇}
\DefineEmoji{speaker-low-volume}{🔈}
\DefineEmoji{speaker-medium-volume}{🔉}
\DefineEmoji{speaker-high-volume}{🔊}
\DefineEmoji{loudspeaker}{📢}
\DefineEmoji{megaphone}{📣}
\DefineEmoji{postal-horn}{📯}
\DefineEmoji{bell}{🔔}
\DefineEmoji{bell-with-slash}{🔕}
\EmojiEndSubgroup{objects}{sound}

\EmojiBeginSubgroup{objects}{music}
\DefineEmoji{musical-score}{🎼}
\DefineEmoji{musical-note}{🎵}
\DefineEmoji{musical-notes}{🎶}
\DefineEmoji{studio-microphone}{🎙️}
\DefineEmoji{level-slider}{🎚️}
\DefineEmoji{control-knobs}{🎛️}
\DefineEmoji{microphone}{🎤}
\DefineEmoji{headphone}{🎧}
\DefineEmoji{radio}{📻}
\EmojiEndSubgroup{objects}{music}

\EmojiBeginSubgroup{objects}{musical-instrument}
\DefineEmoji{saxophone}{🎷}
\DefineEmoji{accordion}{🪗}
\DefineEmoji{guitar}{🎸}
\DefineEmoji{musical-keyboard}{🎹}
\DefineEmoji{trumpet}{🎺}
\DefineEmoji{violin}{🎻}
\DefineEmoji{banjo}{🪕}
\DefineEmoji{drum}{🥁}
\DefineEmoji{long-drum}{🪘}
\DefineEmoji{maracas}{🪇}
\DefineEmoji{flute}{🪈}
\EmojiEndSubgroup{objects}{musical-instrument}

\EmojiBeginSubgroup{objects}{phone}
\DefineEmoji{mobile-phone}{📱}
\DefineEmoji{mobile-phone-with-arrow}{📲}
\DefineEmoji{telephone}{☎️}
\DefineEmoji{telephone-receiver}{📞}
\DefineEmoji{pager}{📟}
\DefineEmoji{fax-machine}{📠}
\EmojiEndSubgroup{objects}{phone}

\EmojiBeginSubgroup{objects}{computer}
\DefineEmoji{battery}{🔋}
\DefineEmoji{low-battery}{🪫}
\DefineEmoji{electric-plug}{🔌}
\DefineEmoji{laptop}{💻}
\DefineEmoji{desktop-computer}{🖥️}
\DefineEmoji{printer}{🖨️}
\DefineEmoji{keyboard}{⌨️}
\DefineEmoji{computer-mouse}{🖱️}
\DefineEmoji{trackball}{🖲️}
\DefineEmoji{computer-disk}{💽}
\DefineEmoji{floppy-disk}{💾}
\DefineEmoji{optical-disk}{💿}
\DefineEmoji{dvd}{📀}
\DefineEmoji{abacus}{🧮}
\EmojiEndSubgroup{objects}{computer}

\EmojiBeginSubgroup{objects}{light-and-video}
\DefineEmoji{movie-camera}{🎥}
\DefineEmoji{film-frames}{🎞️}
\DefineEmoji{film-projector}{📽️}
\DefineEmoji{clapper-board}{🎬}
\DefineEmoji{television}{📺}
\DefineEmoji{camera}{📷}
\DefineEmoji{camera-with-flash}{📸}
\DefineEmoji{video-camera}{📹}
\DefineEmoji{videocassette}{📼}
\DefineEmoji{loupe-left}{🔍}
\DefineEmoji{loupe-right}{🔎}
\DefineEmoji{candle}{🕯️}
\DefineEmoji{light-bulb}{💡}
\DefineEmoji{flashlight}{🔦}
\DefineEmoji{red-paper-lantern}{🏮}
\DefineEmoji{diya-lamp}{🪔}
\EmojiEndSubgroup{objects}{light-and-video}

\EmojiBeginSubgroup{objects}{book-paper}
\DefineEmoji{notebook-with-decorative-cover}{📔}
\DefineEmoji{closed-book}{📕}
\DefineEmoji{open-book}{📖}
\DefineEmoji{green-book}{📗}
\DefineEmoji{blue-book}{📘}
\DefineEmoji{orange-book}{📙}
\DefineEmoji{books}{📚}
\DefineEmoji{notebook}{📓}
\DefineEmoji{ledger}{📒}
\DefineEmoji{page-with-curl}{📃}
\DefineEmoji{scroll}{📜}
\DefineEmoji{page-facing-up}{📄}
\DefineEmoji{newspaper}{📰}
\DefineEmoji{rolled-up-newspaper}{🗞️}
\DefineEmoji{bookmark-tabs}{📑}
\DefineEmoji{bookmark}{🔖}
\DefineEmoji{label}{🏷️}
\EmojiEndSubgroup{objects}{book-paper}

\EmojiBeginSubgroup{objects}{money}
\DefineEmoji{money-bag}{💰}
\DefineEmoji{coin}{🪙}
\DefineEmoji{yen-banknote}{💴}
\DefineEmoji{dollar-banknote}{💵}
\DefineEmoji{euro-banknote}{💶}
\DefineEmoji{pound-banknote}{💷}
\DefineEmoji{money-with-wings}{💸}
\DefineEmoji{credit-card}{💳}
\DefineEmoji{receipt}{🧾}
\DefineEmoji{chart-increasing-with-yen}{💹}
\EmojiEndSubgroup{objects}{money}

\EmojiBeginSubgroup{objects}{mail}
\DefineEmoji{envelope}{✉️}
\DefineEmoji{e-mail}{📧}
\DefineEmoji{incoming-envelope}{📨}
\DefineEmoji{envelope-with-arrow}{📩}
\DefineEmoji{outbox-tray}{📤}
\DefineEmoji{inbox-tray}{📥}
\DefineEmoji{package}{📦}
\DefineEmoji{closed-mailbox-with-raised-flag}{📫}
\DefineEmoji{closed-mailbox-with-lowered-flag}{📪}
\DefineEmoji{open-mailbox-with-raised-flag}{📬}
\DefineEmoji{open-mailbox-with-lowered-flag}{📭}
\DefineEmoji{postbox}{📮}
\DefineEmoji{ballot-box-with-ballot}{🗳️}
\EmojiEndSubgroup{objects}{mail}

\EmojiBeginSubgroup{objects}{writing}
\DefineEmoji{pencil}{✏️}
\DefineEmoji{black-nib}{✒️}
\DefineEmoji{fountain-pen}{🖋️}
\DefineEmoji{pen}{🖊️}
\DefineEmoji{paintbrush}{🖌️}
\DefineEmoji{crayon}{🖍️}
\DefineEmoji{memo}{📝}
\EmojiEndSubgroup{objects}{writing}

\EmojiBeginSubgroup{objects}{office}
\DefineEmoji{briefcase}{💼}
\DefineEmoji{file-folder}{📁}
\DefineEmoji{open-file-folder}{📂}
\DefineEmoji{card-index-dividers}{🗂️}
\DefineEmoji{calendar}{📅}
\DefineEmoji{tear-off-calendar}{📆}
\DefineEmoji{spiral-notepad}{🗒️}
\DefineEmoji{spiral-calendar}{🗓️}
\DefineEmoji{card-index}{📇}
\DefineEmoji{chart-increasing}{📈}
\DefineEmoji{chart-decreasing}{📉}
\DefineEmoji{bar-chart}{📊}
\DefineEmoji{clipboard}{📋}
\DefineEmoji{pushpin}{📌}
\DefineEmoji{round-pushpin}{📍}
\DefineEmoji{paperclip}{📎}
\DefineEmoji{linked-paperclips}{🖇️}
\DefineEmoji{straight-ruler}{📏}
\DefineEmoji{triangular-ruler}{📐}
\DefineEmoji{scissors}{✂️}
\DefineEmoji{card-file-box}{🗃️}
\DefineEmoji{file-cabinet}{🗄️}
\DefineEmoji{wastebasket}{🗑️}
\EmojiEndSubgroup{objects}{office}

\EmojiBeginSubgroup{objects}{lock}
\DefineEmoji{locked}{🔒}
\DefineEmoji{unlocked}{🔓}
\DefineEmoji{locked-with-pen}{🔏}
\DefineEmoji{locked-with-key}{🔐}
\DefineEmoji{key}{🔑}
\DefineEmoji{old-key}{🗝️}
\EmojiEndSubgroup{objects}{lock}

\EmojiBeginSubgroup{objects}{tool}
\DefineEmoji{hammer}{🔨}
\DefineEmoji{axe}{🪓}
\DefineEmoji{pick}{⛏️}
\DefineEmoji{hammer-and-pick}{⚒️}
\DefineEmoji{hammer-and-wrench}{🛠️}
\DefineEmoji{dagger}{🗡️}
\DefineEmoji{crossed-swords}{⚔️}
\DefineEmoji{bomb}{💣}
\DefineEmoji{boomerang}{🪃}
\DefineEmoji{bow-and-arrow}{🏹}
\DefineEmoji{shield}{🛡️}
\DefineEmoji{carpentry-saw}{🪚}
\DefineEmoji{wrench}{🔧}
\DefineEmoji{screwdriver}{🪛}
\DefineEmoji{nut-and-bolt}{🔩}
\DefineEmoji{gear}{⚙️}
\DefineEmoji{clamp}{🗜️}
\DefineEmoji{balance-scale}{⚖️}
\DefineEmoji{white-cane}{🦯}
\DefineEmoji{link}{🔗}
\DefineEmoji{chains}{⛓️}
\DefineEmoji{hook}{🪝}
\DefineEmoji{toolbox}{🧰}
\DefineEmoji{magnet}{🧲}
\DefineEmoji{ladder}{🪜}
\EmojiEndSubgroup{objects}{tool}

\EmojiBeginSubgroup{objects}{science}
\DefineEmoji{alembic}{⚗️}
\DefineEmoji{test-tube}{🧪}
\DefineEmoji{petri-dish}{🧫}
\DefineEmoji{dna}{🧬}
\DefineEmoji{microscope}{🔬}
\DefineEmoji{telescope}{🔭}
\DefineEmoji{satellite-antenna}{📡}
\EmojiEndSubgroup{objects}{science}

\EmojiBeginSubgroup{objects}{medical}
\DefineEmoji{syringe}{💉}
\DefineEmoji{drop-of-blood}{🩸}
\DefineEmoji{pill}{💊}
\DefineEmoji{adhesive-bandage}{🩹}
\DefineEmoji{crutch}{🩼}
\DefineEmoji{stethoscope}{🩺}
\DefineEmoji{x-ray}{🩻}
\EmojiEndSubgroup{objects}{medical}

\EmojiBeginSubgroup{objects}{household}
\DefineEmoji{door}{🚪}
\DefineEmoji{elevator}{🛗}
\DefineEmoji{mirror}{🪞}
\DefineEmoji{window}{🪟}
\DefineEmoji{bed}{🛏️}
\DefineEmoji{couch-and-lamp}{🛋️}
\DefineEmoji{chair}{🪑}
\DefineEmoji{toilet}{🚽}
\DefineEmoji{plunger}{🪠}
\DefineEmoji{shower}{🚿}
\DefineEmoji{bathtub}{🛁}
\DefineEmoji{mouse-trap}{🪤}
\DefineEmoji{razor}{🪒}
\DefineEmoji{lotion-bottle}{🧴}
\DefineEmoji{safety-pin}{🧷}
\DefineEmoji{broom}{🧹}
\DefineEmoji{basket}{🧺}
\DefineEmoji{roll-of-paper}{🧻}
\DefineEmoji{bucket}{🪣}
\DefineEmoji{soap}{🧼}
\DefineEmoji{bubbles}{🫧}
\DefineEmoji{toothbrush}{🪥}
\DefineEmoji{sponge}{🧽}
\DefineEmoji{fire-extinguisher}{🧯}
\DefineEmoji{shopping-cart}{🛒}
\EmojiEndSubgroup{objects}{household}

\EmojiBeginSubgroup{objects}{other-object}
\DefineEmoji{cigarette}{🚬}
\DefineEmoji{coffin}{⚰️}
\DefineEmoji{headstone}{🪦}
\DefineEmoji{funeral-urn}{⚱️}
\DefineEmoji{nazar-amulet}{🧿}
\DefineEmoji{hamsa}{🪬}
\DefineEmoji{moai}{🗿}
\DefineEmoji{placard}{🪧}
\DefineEmoji{identification-card}{🪪}
\EmojiEndSubgroup{objects}{other-object}
\EmojiEndGroup{objects}


\EmojiBeginGroup{symbols}
\EmojiBeginSubgroup{symbols}{transport-sign}
\DefineEmoji{atm-sign}{🏧}
\DefineEmoji{litter-in-bin-sign}{🚮}
\DefineEmoji{potable-water}{🚰}
\DefineEmoji{wheelchair-symbol}{♿}
\DefineEmoji{mens-room}{🚹}
\DefineEmoji{womens-room}{🚺}
\DefineEmoji{restroom}{🚻}
\DefineEmoji{baby-symbol}{🚼}
\DefineEmoji{water-closet}{🚾}
\DefineEmoji{passport-control}{🛂}
\DefineEmoji{customs}{🛃}
\DefineEmoji{baggage-claim}{🛄}
\DefineEmoji{left-luggage}{🛅}
\EmojiEndSubgroup{symbols}{transport-sign}

\EmojiBeginSubgroup{symbols}{warning}
\DefineEmoji{warning}{⚠️}
\DefineEmoji{children-crossing}{🚸}
\DefineEmoji{no-entry}{⛔}
\DefineEmoji{prohibited}{🚫}
\DefineEmoji{no-bicycles}{🚳}
\DefineEmoji{no-smoking}{🚭}
\DefineEmoji{no-littering}{🚯}
\DefineEmoji{non-potable-water}{🚱}
\DefineEmoji{no-pedestrians}{🚷}
\DefineEmoji{no-mobile-phones}{📵}
\DefineEmoji{no-one-under-eighteen}{🔞}
\DefineEmoji{radioactive}{☢️}
\DefineEmoji{biohazard}{☣️}
\EmojiEndSubgroup{symbols}{warning}

\EmojiBeginSubgroup{symbols}{arrow}
\DefineEmoji{up-arrow}{⬆️}
\DefineEmoji{up-right-arrow}{↗️}
\DefineEmoji{right-arrow}{➡️}
\DefineEmoji{down-right-arrow}{↘️}
\DefineEmoji{down-arrow}{⬇️}
\DefineEmoji{down-left-arrow}{↙️}
\DefineEmoji{left-arrow}{⬅️}
\DefineEmoji{up-left-arrow}{↖️}
\DefineEmoji{up-down-arrow}{↕️}
\DefineEmoji{left-right-arrow}{↔️}
\DefineEmoji{right-arrow-curving-left}{↩️}
\DefineEmoji{left-arrow-curving-right}{↪️}
\DefineEmoji{right-arrow-curving-up}{⤴️}
\DefineEmoji{right-arrow-curving-down}{⤵️}
\DefineEmoji{clockwise-vertical-arrows}{🔃}
\DefineEmoji{counterclockwise-arrows-button}{🔄}
\DefineEmoji{back-arrow}{🔙}
\DefineEmoji{end-arrow}{🔚}
\DefineEmoji{on-arrow}{🔛}
\DefineEmoji{soon-arrow}{🔜}
\DefineEmoji{top-arrow}{🔝}
\EmojiEndSubgroup{symbols}{arrow}

\EmojiBeginSubgroup{symbols}{religion}
\DefineEmoji{place-of-worship}{🛐}
\DefineEmoji{atom-symbol}{⚛️}
\DefineEmoji{om}{🕉️}
\DefineEmoji{star-of-david}{✡️}
\DefineEmoji{wheel-of-dharma}{☸️}
\DefineEmoji{yin-yang}{☯️}
\DefineEmoji{latin-cross}{✝️}
\DefineEmoji{orthodox-cross}{☦️}
\DefineEmoji{star-and-crescent}{☪️}
\DefineEmoji{peace-symbol}{☮️}
\DefineEmoji{menorah}{🕎}
\DefineEmoji{dotted-six-pointed-star}{🔯}
\DefineEmoji{khanda}{🪯}
\EmojiEndSubgroup{symbols}{religion}

\EmojiBeginSubgroup{symbols}{zodiac}
\DefineEmoji{aries}{♈}
\DefineEmoji{taurus}{♉}
\DefineEmoji{gemini}{♊}
\DefineEmoji{cancer}{♋}
\DefineEmoji{leo}{♌}
\DefineEmoji{virgo}{♍}
\DefineEmoji{libra}{♎}
\DefineEmoji{scorpio}{♏}
\DefineEmoji{sagittarius}{♐}
\DefineEmoji{capricorn}{♑}
\DefineEmoji{aquarius}{♒}
\DefineEmoji{pisces}{♓}
\DefineEmoji{ophiuchus}{⛎}
\EmojiEndSubgroup{symbols}{zodiac}

\EmojiBeginSubgroup{symbols}{av-symbol}
\DefineEmoji{shuffle-tracks-button}{🔀}
\DefineEmoji{repeat-button}{🔁}
\DefineEmoji{repeat-single-button}{🔂}
\DefineEmoji{play-button}{▶️}
\DefineEmoji{fast-forward-button}{⏩}
\DefineEmoji{next-track-button}{⏭️}
\DefineEmoji{play-or-pause-button}{⏯️}
\DefineEmoji{reverse-button}{◀️}
\DefineEmoji{fast-reverse-button}{⏪}
\DefineEmoji{last-track-button}{⏮️}
\DefineEmoji{upwards-button}{🔼}
\DefineEmoji{fast-up-button}{⏫}
\DefineEmoji{downwards-button}{🔽}
\DefineEmoji{fast-down-button}{⏬}
\DefineEmoji{pause-button}{⏸️}
\DefineEmoji{stop-button}{⏹️}
\DefineEmoji{record-button}{⏺️}
\DefineEmoji{eject-button}{⏏️}
\DefineEmoji{cinema}{🎦}
\DefineEmoji{dim-button}{🔅}
\DefineEmoji{bright-button}{🔆}
\DefineEmoji{antenna-bars}{📶}
\DefineEmoji{wireless}{🛜}
\DefineEmoji{vibration-mode}{📳}
\DefineEmoji{mobile-phone-off}{📴}
\EmojiEndSubgroup{symbols}{av-symbol}

\EmojiBeginSubgroup{symbols}{gender}
\DefineEmoji{female-sign}{♀️}
\DefineEmoji{male-sign}{♂️}
\DefineEmoji{transgender-symbol}{⚧️}
\EmojiEndSubgroup{symbols}{gender}

\EmojiBeginSubgroup{symbols}{math}
\DefineEmoji{multiply}{✖️}
\DefineEmoji{plus}{➕}
\DefineEmoji{minus}{➖}
\DefineEmoji{divide}{➗}
\DefineEmoji{heavy-equals-sign}{🟰}
\DefineEmoji{infinity}{♾️}
\EmojiEndSubgroup{symbols}{math}

\EmojiBeginSubgroup{symbols}{punctuation}
\DefineEmoji{double-exclamation-mark}{‼️}
\DefineEmoji{exclamation-question-mark}{⁉️}
\DefineEmoji{red-question-mark}{❓}
\DefineEmoji{white-question-mark}{❔}
\DefineEmoji{white-exclamation-mark}{❕}
\DefineEmoji{red-exclamation-mark}{❗}
\DefineEmoji{wavy-dash}{〰️}
\EmojiEndSubgroup{symbols}{punctuation}

\EmojiBeginSubgroup{symbols}{currency}
\DefineEmoji{currency-exchange}{💱}
\DefineEmoji{heavy-dollar-sign}{💲}
\EmojiEndSubgroup{symbols}{currency}

\EmojiBeginSubgroup{symbols}{other-symbol}
\DefineEmoji{medical-symbol}{⚕️}
\DefineEmoji{recycling-symbol}{♻️}
\DefineEmoji{fleur-de-lis}{⚜️}
\DefineEmoji{trident-emblem}{🔱}
\DefineEmoji{name-badge}{📛}
\DefineEmoji{japanese-symbol-for-beginner}{🔰}
\DefineEmoji{hollow-red-circle}{⭕}
\DefineEmoji{check-mark-button}{✅}
\DefineEmoji{check-box-with-check}{☑️}
\DefineEmoji{check-mark}{✔️}
\DefineEmoji{cross-mark}{❌}
\DefineEmoji{cross-mark-button}{❎}
\DefineEmoji{curly-loop}{➰}
\DefineEmoji{double-curly-loop}{➿}
\DefineEmoji{part-alternation-mark}{〽️}
\DefineEmoji{eight-spoked-asterisk}{✳️}
\DefineEmoji{eight-pointed-star}{✴️}
\DefineEmoji{sparkle}{❇️}
\DefineEmoji{copyright}{©️}
\DefineEmoji{registered}{®️}
\DefineEmoji{trade-mark}{™️}
\EmojiEndSubgroup{symbols}{other-symbol}

\EmojiBeginSubgroup{symbols}{keycap}
\DefineEmoji{keycap-hash}{\char"0023\char"FE0F\char"20E3}
\DefineEmoji{keycap-star}{*️⃣}
\DefineEmoji{keycap-zero}{0️⃣}
\DefineEmoji{keycap-one}{1️⃣}
\DefineEmoji{keycap-two}{2️⃣}
\DefineEmoji{keycap-three}{3️⃣}
\DefineEmoji{keycap-four}{4️⃣}
\DefineEmoji{keycap-five}{5️⃣}
\DefineEmoji{keycap-six}{6️⃣}
\DefineEmoji{keycap-seven}{7️⃣}
\DefineEmoji{keycap-eight}{8️⃣}
\DefineEmoji{keycap-nine}{9️⃣}
\DefineEmoji{keycap-ten}{🔟}
\EmojiEndSubgroup{symbols}{keycap}

\EmojiBeginSubgroup{symbols}{alphanum}
\DefineEmoji{input-latin-uppercase}{🔠}
\DefineEmoji{input-latin-lowercase}{🔡}
\DefineEmoji{input-numbers}{🔢}
\DefineEmoji{input-symbols}{🔣}
\DefineEmoji{input-latin-letters}{🔤}
\DefineEmoji{a-button}{🅰️}
\DefineEmoji{ab-button}{🆎}
\DefineEmoji{b-button}{🅱️}
\DefineEmoji{cl-button}{🆑}
\DefineEmoji{cool-button}{🆒}
\DefineEmoji{free-button}{🆓}
\DefineEmoji{information}{ℹ️}
\DefineEmoji{id-button}{🆔}
\DefineEmoji{circled-m}{Ⓜ️}
\DefineEmoji{new-button}{🆕}
\DefineEmoji{ng-button}{🆖}
\DefineEmoji{o-button}{🅾️}
\DefineEmoji{ok-button}{🆗}
\DefineEmoji{p-button}{🅿️}
\DefineEmoji{sos-button}{🆘}
\DefineEmoji{up-button}{🆙}
\DefineEmoji{vs-button}{🆚}
\DefineEmoji{japanese-here-button}{🈁}
\DefineEmoji{japanese-service-charge-button}{🈂️}
\DefineEmoji{japanese-monthly-amount-button}{🈷️}
\DefineEmoji{japanese-not-free-of-charge-button}{🈶}
\DefineEmoji{japanese-reserved-button}{🈯}
\DefineEmoji{japanese-bargain-button}{🉐}
\DefineEmoji{japanese-discount-button}{🈹}
\DefineEmoji{japanese-free-of-charge-button}{🈚}
\DefineEmoji{japanese-prohibited-button}{🈲}
\DefineEmoji{japanese-acceptable-button}{🉑}
\DefineEmoji{japanese-application-button}{🈸}
\DefineEmoji{japanese-passing-grade-button}{🈴}
\DefineEmoji{japanese-vacancy-button}{🈳}
\DefineEmoji{japanese-congratulations-button}{㊗️}
\DefineEmoji{japanese-secret-button}{㊙️}
\DefineEmoji{japanese-open-for-business-button}{🈺}
\DefineEmoji{japanese-no-vacancy-button}{🈵}
\EmojiEndSubgroup{symbols}{alphanum}

\EmojiBeginSubgroup{symbols}{geometric}
\DefineEmoji{red-circle}{🔴}
\DefineEmoji{orange-circle}{🟠}
\DefineEmoji{yellow-circle}{🟡}
\DefineEmoji{green-circle}{🟢}
\DefineEmoji{blue-circle}{🔵}
\DefineEmoji{purple-circle}{🟣}
\DefineEmoji{brown-circle}{🟤}
\DefineEmoji{black-circle}{⚫}
\DefineEmoji{white-circle}{⚪}
\DefineEmoji{red-square}{🟥}
\DefineEmoji{orange-square}{🟧}
\DefineEmoji{yellow-square}{🟨}
\DefineEmoji{green-square}{🟩}
\DefineEmoji{blue-square}{🟦}
\DefineEmoji{purple-square}{🟪}
\DefineEmoji{brown-square}{🟫}
\DefineEmoji{black-large-square}{⬛}
\DefineEmoji{white-large-square}{⬜}
\DefineEmoji{black-medium-square}{◼️}
\DefineEmoji{white-medium-square}{◻️}
\DefineEmoji{black-medium-small-square}{◾}
\DefineEmoji{white-medium-small-square}{◽}
\DefineEmoji{black-small-square}{▪️}
\DefineEmoji{white-small-square}{▫️}
\DefineEmoji{large-orange-diamond}{🔶}
\DefineEmoji{large-blue-diamond}{🔷}
\DefineEmoji{small-orange-diamond}{🔸}
\DefineEmoji{small-blue-diamond}{🔹}
\DefineEmoji{red-triangle-pointed-up}{🔺}
\DefineEmoji{red-triangle-pointed-down}{🔻}
\DefineEmoji{diamond-with-a-dot}{💠}
\DefineEmoji{radio-button}{🔘}
\DefineEmoji{white-square-button}{🔳}
\DefineEmoji{black-square-button}{🔲}
\EmojiEndSubgroup{symbols}{geometric}
\EmojiEndGroup{symbols}


\EmojiBeginGroup{flags}
\EmojiBeginSubgroup{flags}{flag}
\DefineEmoji{chequered-flag}{🏁}
\DefineEmoji{triangular-flag}{🚩}
\DefineEmoji{crossed-flags}{🎌}
\DefineEmoji{black-flag}{🏴}
\DefineEmoji{white-flag}{🏳️}
\DefineEmoji{rainbow-flag}{🏳️‍🌈}
\DefineEmoji{transgender-flag}{🏳️‍⚧️}
\DefineEmoji{pirate-flag}{🏴‍☠️}
\EmojiEndSubgroup{flags}{flag}

\EmojiBeginSubgroup{flags}{country-flag}
\DefineEmoji{eu}{🇪🇺}
\EmojiEndSubgroup{flags}{country-flag}
\EmojiEndGroup{flags}

\ifEmojiExtra
\EmojiBeginGroup{extra}
\EmojiBeginSubgroup{extra}{extra}
\DefineEmoji{lingchi}{凌遲}
\DefineEmoji{YHWH}{יהוה}
\EmojiEndSubgroup{extra}{extra}
\EmojiEndGroup{extra}
\fi

%    \end{macrocode}
% \end{macro}
%
%
% ^^A ----------------------------------------------------------------------------------
% \subsection{Internal Macros}
%
% \begin{macro}{\emo@error@fg}
% \begin{macro}{\emo@error@bg}
% \begin{macro}{\emo@error}
% Define two colors and a function that uses the two colors for formatting an
% attention-grabbing error message. If you use an invalid emoji name and
% overlook the warning in the console, you \emph{will} notice the error messsage
% in the document thusly formatted.
%    \begin{macrocode}
\definecolor{emo@error@fg}{rgb}{1,1,1}
\definecolor{emo@error@bg}{rgb}{.6824,.0863,.0863}
\def\emo@error#1{%
    \colorbox{emo@error@bg}{%
        \textcolor{emo@error@fg}{%
            \textsf{Bad} \texttt{\textbackslash emo\{#1\}}%
        }%
    }%
}
%    \end{macrocode}
% \end{macro}
% \end{macro}
% \end{macro}
%
% \begin{macro}{\emo@ifdef}
% Validate the emoji name given as first argument. The macro expands to the
% second argument if the name is valid and an error message otherwise. Its
% implementation relies on the |emo@emoji| table.
%    \begin{macrocode}
\def\emo@ifdef#1#2{%
    \ifcsname emo@emoji@#1\endcsname#2\else%
        \PackageWarning{emo}{Unknown emoji name in `\string\emo{#1}'}%
        \emo@error{#1}%
    \fi%
}
%    \end{macrocode}
% \end{macro}
%
% \begin{macro}{\emo@index}
% If indexing is enabled, record the use of an emoji. Otherwise, do nothing.
%    \begin{macrocode}
\ifemo@indexing
    \newindex{emo}{edx}{end}{Emoji Index}
    \def\emo@index#1{\index[emo]{#1}}
\else
    \def\emo@index#1{}
\fi
%    \end{macrocode}
% \end{macro}
%
% \begin{macro}{\emo@content}
% Emit the emoji content.
%    \begin{macrocode}
\ifx\emo@backend\emo@use@unicode
    \def\emo@content#1{\begingroup\csname emo@emoji@#1\endcsname\endgroup}
\else
\ifx\emo@backend\emo@use@font
    \newfontface\emo@font[Renderer=Harfbuzz]{NotoColorEmoji.ttf}
    \def\emo@content#1{\begingroup\emo@font\csname emo@emoji@#1\endcsname\endgroup}
\else
    \def\emo@content#1{%
        \raisebox{-0.2ex}{%
            \includegraphics[height=1em]{emo-graphics/emo-#1}}
    }
\fi
\fi
%    \end{macrocode}
% \end{macro}
%
%
% ^^A ----------------------------------------------------------------------------------
% \subsection{User Macros}
%
% \begin{macro}{\emo}
% Thanks to the above macros, the main |\emo| macro is really simple: If the
% emoji name is defined, emit an index entry and the emoji content.
%    \begin{macrocode}
\newcommand\emo[1]{%
    \emo@ifdef{#1}{%
        \emo@index{#1}%
        \emo@content{#1}%
    }%
}
%    \end{macrocode}
% \end{macro}
%
% \begin{macro}{\lingchi}
% \begin{macro}{\YHWH}
% By default, the implementations for |\lingchi| and |\YHWH| simply delegate to
% |\emo|. After all, both ``emoji'' also appear in the emoji table and have PDF
% graphics. While that works for the Unicode and PDF backends, it does
% \emph{not} for the fonts backend, since |\emo| uses the wrong font. So we define
% different versions that use the correct fonts.
%    \begin{macrocode}
\ifemo@extra
\ifx\emo@backend\emo@use@font
    \newfontface\emo@chinese{emo-lingchi.ttf}
    \newfontface\emo@hebrew{LinLibertine_R.otf}
    \newcommand\lingchi{%
        \emo@index{lingchi}%
        \begingroup\emo@chinese\emo@emoji@lingchi\endgroup%
        \xspace}
    \newcommand\YHWH{%
        \emo@index{YHWH}%
        \begingroup\emo@hebrew\emo@emoji@YHWH\endgroup%
        \xspace}
\else
    \newcommand\lingchi{\emo{lingchi}\xspace}
    \newcommand\YHWH{\emo{YHWH}\xspace}
\fi
\fi
%    \end{macrocode}
% \end{macro}
% \end{macro}
%
% Et voil\`a. That's it!
%
%    \begin{macrocode}
%</package>
%    \end{macrocode}
%
%
% ^^A ==================================================================================
% \section{LaTeXML Binding}
% \changes{0.2}{}{Add LaTeXML binding}
%
% To support conversion from LaTeX to HTML, emo includes a so-called binding for
% \href{https://github.com/brucemiller/LaTeXML}{LaTeXML}. It effectively is a
% (much simplified) re-implementation of emo's core functionality, only written
% in Perl against LaTeXML's API. The binding ignores the \textsf{index} option
% and does not perform error checking on emoji names. If either is important to
% you, please compile the document with LaTeX first. Furthermore, the binding
% emits necessary Unicode codepoints only, without font annotations. If you want
% to specify fonts, please use a CSS fontstack.
%
% Asking package authors to reimplement their packages for LaTeXML seems
% unreasonable to me. It leads to code duplication and places the maintenance
% burden on package authors. Yet, right after announcing emo, the question of
% LaTeXML support came up. LaTeXML includes the |latexml.sty| package, which
% defines |\iflatexml|. I would have used that command to make the three-line
% change to |emo.sty| necessary to support LaTeXML, except |latexml.sty|
% contains lots of other stuff that isn't needed. Always loading lots of macros
% only to detect LaTeXML slows down compilation and wastes memory. Since
% reimplementing |\iflatexml| would require a binding anyways, I just wrote a
% minimal binding. As I said, LaTeXML's approach is broken.
%
% With that out of the way, let's get started:
%    \begin{macrocode}
%<*latexml-binding>
%    \end{macrocode}
%
% The binding starts with an explicit preamble because |docstrip| does not
% alllow for a redefinition of the starting characters of a line comment. It is
% followed by the Perl dependencies.
%    \begin{macrocode}
## emo's LaTeXML binding.
## (C) 2023 by Robert Grimm.
## Released under LPPL v1.3c or later.
use strict;
use warnings;
use LaTeXML::Package;
%    \end{macrocode}
%
% \begin{macro}{\ifemo@extra}
% Next, we use raw TeX to declare the LaTeX package and define the |emo@extra|
% conditional. There is no need to define the |emo@indexing| conditional, since
% it corresponds to the unsupported \textsf{index} option.
%    \begin{macrocode}
RawTeX(<<'EOTeX');
\ProvidesPackage{emo}
    [2023/04/21 v0.3 emo•ji for all (LaTeX engines)]
\newif\ifemo@extra\emo@extrafalse
EOTeX
%    \end{macrocode}
% \end{macro}
%
% Option prcessing is almost trivial:
%    \begin{macrocode}
DeclareOption('extra', '\emo@extratrue');
DeclareOption('index', '');
ProcessOptions();
%    \end{macrocode}
%
% \begin{macro}{\emo@emoji@<name>}
% \begin{macro}{\emo}
% Just like the actual package implementation, the binding reads the emoji table
% from |emo.def|. Similar to the actual implementation of the |\emo| macro when
% running under LuaLaTeX, the binding expands the named entry from the emoji
% table, producing the emoji's Unicode codepoints.
%    \begin{macrocode}
InputDefinitions('emo', type => 'def', noltxml => 1);
DefMacro('\emo{}', '\csname emo@emoji@#1\endcsname');
%    \end{macrocode}
% \end{macro}
% \end{macro}
%
% \begin{macro}{\lingchi}
% \begin{macro}{\YHWH}
% If the |emo@extra| conditional is enabled, require the |xspace| package and then
% provide minimal re-definitions of the |\lingchi| and |\YHWH| macros. Both simply
% expand to the necessary Unicode codepoints.
%    \begin{macrocode}
if (IfCondition(T_CS('\ifemo@extra'))) {
    RequirePackage('xspace');
    DefMacro('\lingchi', "\x{51cc}\x{9072}\\xspace");
    DefMacro('\YHWH', "\x{05D9}\x{05D4}\x{05D5}\x{05D4}\\xspace");
}
%    \end{macrocode}
% \end{macro}
% \end{macro}
%
% That's it for the binding, too.
%    \begin{macrocode}
%</latexml-binding>
%    \end{macrocode}
%
%
% \Finale
